\section{VINS预积分推导}
\subsection{记号说明}
\subsubsection{上下标}
\par b = body frame = imu frame
\par w = world frame

\subsection{IMU noise model}
\begin{equation}
	\begin{split}
		\hat{a}_t &= a_t + R_w^t g^w + b_{a_t} + n_a\\
		&= R_w^t (a_t^w+g^w)+b_{a_t}+n_{a}\\
		\hat{\omega}_t &= \omega_t+b_{gt}+n_{g}
	\end{split}
\end{equation}
$\hat{a}_t$ is in IMU frame and $a_t^w$ is in world frame.

\subsection{Preintegration on continuous time}
In world frame, kinetic equations can be written as follows:
\begin{equation}
	\begin{split}
		p_{b_{k+1}}^w & = p_{b_k}^w + v_{b_k}^w \Delta t_k +\iint_{t\in [t_k,t_{k+1}]} (R^w_t(\hat{a}_t-b_{a_t}-n_a)-g^w) dt^2 \\
		v_{b+1}^w & = v_{b_k}^w + \int_{t\in [t_k,t_{k+1}]} (R^w_t(\hat{a}_t-b_{a_t}-n_a)-g^w) dt \\
		q_{b_{k+1}}^w & = q_{b_k}^w \otimes \int_{t\in [t_k,t_{k+1}]} \frac{1}{2} \Omega(\hat{w}_t-b_{g_t}-n_g)q_t^{b_k} dt
	\end{split}
\end{equation}
\par In IMU frame w.r.t. previous time,
\begin{equation} \label{eq:vins-preintegration}
	\begin{split}
		R_w^{b_k}p_{b_{k+1}}^w &= R_w^{b_k}( p_{b_k}^w + v_{b_k}^w\Delta{t} - \frac12g^w\Delta{t}^2) + \alpha_{b_{k+1}}^{b_k} \\
		R_w^{b_k}v_{b_{k+1}}^w &= R_w^{b_k}( v_{b_k}^w - g^w\Delta{t} ) + \beta_{b_{k+1}}^{b_k} \\
		q_{b_{k+1}}^w &= q_{b_k}^w \otimes \gamma_{b_{k+1}}^{b_k}
	\end{split}
\end{equation}
\par where contains preintegration quantity $\alpha_{b_{k+1}}^{b_k}$, $\beta_{b_{k+1}}^{b_k}$ and $\gamma_{b_{k+1}}^{b_k}$.
\begin{equation}
	\begin{split}
		\alpha_{b_{k+1}}^{b_k} &= \iint_{t\in[t_k,t_{k+1}]}R_t^{b_k}(\hat{a}_t-b_{a_t}-n_a)dt^2 \\
		\beta_{b_{k+1}}^{b_k} &= \int_{t\in[t_k,t_{k+1}]}R_t^{b_k}(\hat{a}_t-b_{a_t}-n_a)dt \\
		\gamma_{b_{k+1}}^{b_k} &= \int_{t\in[t_k,t_{k+1}]}\frac12\Omega(\hat{\omega}_t-b_{g_t}-n_g)q_tdt
	\end{split}
\end{equation}
\par As we see preintegration quantity is only related with IMU bias $b_{a_t}$ and $b_{g_t}$.

In discrete time, preintegration quantity can be iterated as Eq. \ref{eq:discrete-time-preintegration}. Noise $n_{a_t}$ and $n_{g_t}$ are ignored in the equation because they can not be predicted and their integration can be treated as zero during small interval (in fact the integration of is small random walk). $\alpha_i^{b_k}$, $\beta_i^{b_k}$, $\gamma_i^{b_k}$ will be initialized to be 0 or identity rotation respectively.
\begin{equation} \label{eq:discrete-time-preintegration}
	\begin{split}
		\hat{\alpha}_{i+1}^{b_k} &= \hat{\alpha}_i^{b_k} + \frac12(\hat{\beta}_i^{b_k}+\hat{\beta}_{i+1}^{b_k})\delta{t} \\
		\hat{\beta}_{i+1}^{b_k} &= \hat{\beta}_i^{b_k} + \frac12\left[\hat{\gamma}_i^{b_k}(\hat{a}_i-b_{a_i})+\hat{\gamma}_{i+1}^{b_k}(\hat{a}_{i+1}-b_{a_i})\right]\delta{t} \\
		\hat{\gamma}_{i+1}^{b_k} &= \hat{\gamma}_i^{b_k} \otimes
		\begin{bmatrix}
			1 \\ \frac12(\frac{\hat{w}_i+\hat{w}_{i+1}}2-b_{g_i})\delta{t}
		\end{bmatrix}
	\end{split}
\end{equation}
Applying Eq \ref{eq:discrete-time-preintegration} we get preintegration quantity between two consecutive frames: $\alpha_{b_{k+1}}^{b_k}$, $\beta_{b_{k+1}}^{b_k}$ and $\gamma_{b_{k+1}}^{b_k}$. IMU bias between consecutive frame changes quite small so it can be treated as constant between frame $i$ and frame $i+1$.
So preintegration quantity can be expanded in first-order-taylor:
\begin{equation}
	\begin{split}
		\alpha_{b_{k+1}}^{b_k} &= \hat{\alpha}_{b_{k+1}}^{b_k}+J\delta{b_{a_k}}+J\delta{b_{g_k}} \\
		\beta_{b_{k+1}}^{b_k} &= \hat{\beta}_{b_{k+1}}^{b_k}+J\delta{b_{a_k}}+J\delta{b_{g_k}} \\
		\gamma_{b_{k+1}}^{b_k} &= \hat{\gamma}_{b_{k+1}}^{b_k} \otimes
		\begin{bmatrix}
			1 \\\frac12J\delta{b_{g_k}}
		\end{bmatrix}
	\end{split}
\end{equation}

\subsection{Initialization for lidar-imu system}
\par Denote frame of lidar odometry as $c$, frame of imu as $b$.
\subsubsection{Extrinsic estimation}
外参包含$q_b^c$和$p_b^c$,一般来说,$q_b^c$可以初始化时估计,$p_b^c$量测即可。
\begin{equation} \label{eq:vins-ex-estimate}
	q_{b_{k+1}}^{c_k} = q_{c_{k+1}}^{c_k} \otimes q_b^c = q_b^c \otimes q_{b_{k+1}}^{b_k}
\end{equation}
\par 其中,$q_{c_{k+1}}^{c_k}$和$q_{b_{k+1}}^{b_k}$都是计算得到的,时间相对应,可以通过ceres-solver最小化$q_{c_{k+1}}^{c_k}\otimes q_b^c$和$q_b^c \otimes q_{b_{k+1}}^{b_k}$的$angularDistance$求解$q_b^c$,或者由公式\ref{eq:vins-ex-estimate}得:
\begin{equation}
	\left(\mathcal{L}(q_{b_{k+1}}^{b_k}) - \mathcal{R}(q_{c_{k+1}}^{c_k})\right) q_b^c = 0
\end{equation}
\par 解上述方程时,$\mathcal{L}(q_{b_{k+1}}^{b_k}) - \mathcal{R}(q_{c_{k+1}}^{c_k})$的最小特征值对应的特征矩阵即为$q_b^c$。

\subsubsection{Initialize gyro bias}
\begin{equation}
	\begin{split}
		\gamma_{b_{k+1}}^{b_k} &= \hat{\gamma}_{b_{k+1}}^{b_k} \otimes
		\begin{bmatrix}
			1 \\\frac12J\delta{b_{g_k}}
		\end{bmatrix} \\
		&= {q_{b_k}^{c_0}}^{-1} \otimes q_{b_{k+1}}^{c_0}
	\end{split}
\end{equation}
\begin{equation}
	J\delta{b_{g_k}} = 2 \left\lbrace {{}\hat{\gamma}_{b_{k+1}}^{b_k}}^{-1} \otimes {q_{b_k}^{c_0}}^{-1} \otimes q_{b_{k+1}}^{c_0} \right\rbrace_{vec}
\end{equation}
\par 解上述方程得到陀螺仪偏置$\delta{b_{g_k}}$,而后更新$\alpha_{b_{k+1}}^{b_k}$,$\beta_{b_{k+1}}^{b_k}$和$\gamma_{b_{k+1}}^{b_k}$。

\subsubsection{Initialize velocity and gravity}
\begin{equation} \label{eq:vins-transform}
	\begin{split}
		q_{b_k}^{c_0} &= q_{c_k}^{c_0} \otimes q_b^c \\
		p_{b_k}^{c_0} &= p_{c_k}^{c_0} + R_{c_k}^{c_0} p_b^c
	\end{split}
\end{equation}
\par 公式\ref{eq:vins-preintegration}是预积分量在$w$坐标系下的表示,在$c_0$坐标系下可表示为:
\begin{equation}
	\begin{split}
		R_{c_0}^{b_k}p_{b_{k+1}}^{c_0} &= R_{c_0}^{b_k}( p_{b_k}^{c_0} + v_{b_k}^{c_0}\Delta{t} - \frac12g^{c_0}\Delta{t}^2) + \alpha_{b_{k+1}}^{b_k} \\
		R_{c_0}^{b_k}v_{b_{k+1}}^{c_0} &= R_{c_0}^{b_k}( v_{b_k}^{c_0} - g^{c_0}\Delta{t} ) + \beta_{b_{k+1}}^{b_k}
	\end{split}
\end{equation}
\par 待求量为$v_{b_k}^{c_0}$,$v_{b_{k+1}}^{c_0}$和$g^{c_0}$,写成矩阵形式为:
\begin{equation}
	\begin{bmatrix}
		R_{c_0}^{b_k}\Delta{t} & 0              & -\frac12R_{c_0}^{b_k}\Delta{t}^2 \\
		R_{c_0}^{b_k}          & -R_{c_0}^{b_k} & -R_{c_0}^{b_k}\Delta{t}
	\end{bmatrix}
	\begin{bmatrix}
		v_{b_k}^{c_0} \\ v_{b_{k+1}}^{c_0} \\ g^{c_0}
	\end{bmatrix}
	=
	\begin{bmatrix}
		R_{c_0}^{b_k} (p_{b_{k+1}}^{c_0}-p_{b_k}^{c_0}) - \alpha_{b_{k+1}}^{b_k} \\
		\beta_{b_{k+1}}^{b_k}
	\end{bmatrix}
\end{equation}