\section{图像特征描述子}
\subsection{SIFT}
\subsubsection{高斯模糊}
高斯卷积核函数:
\begin{equation}
  G(x,y,\rho)=\frac{1}{2 \pi \sigma^2} exp\left(-\frac{x^2+y^2}{2\sigma^2}\right)
\end{equation}
\par
高斯卷积核的大小根据$\sigma$确定,一般是奇数维的方阵,方阵半径不超过$[3\sigma]+1$(含中心)。
\par 高斯模糊具有线性可分的性质,可以在二维图像上对两个独立的一维空间分别进行计算,可以大大减少运算次数。
\subsubsection{高斯金字塔}
高斯金字塔的构建过程可以分为两步:高斯平滑、降采样。
高斯金字塔共o组、s层,则有
\begin{equation}
  \sigma(s)=\sigma_0 2^{s/S}
\end{equation}
公式中的$\sigma$是尺度空间坐标,$s$是层的坐标,$\sigma_0$是初始尺度,$S$是每组的层数。
\par 高斯金字塔的组内和组间尺度可以归为:
\begin{equation}
  2^{i-1} (\rho,k\rho,k^2\rho,...,k^{n-1}\rho),
  k=2^{1/S}
\end{equation}
\subsubsection{关键点检测:DoG}
高斯拉普拉斯算子
\begin{equation}
  LOG(x,y,\rho)=\rho^2 \nabla^2 G
\end{equation}
高斯差分算子
\begin{equation}
  D(x,y,\rho)=\left(G(x,y,k\rho)-G(x,y,\rho)\right)*I(x,y)
\end{equation}
\subsubsection{DoG的局部极值检测}

\subsubsection{关键点方向分配}

\subsubsection{关键点描述}
