\section{loam 算法的李代数形式推导}
\newcommand{\vecba}[1]{\boldmath\overrightarrow{#1}\unboldmath}
loam是CMU张博士提出的一种基于特征的ICP算法,主要使用线特征和面特征两种。
\subsection{记号和推导说明}
为了简化推导,本文主要使用三种记号: $+,\wedge,\odot$,意义如下:
\begin{equation}
	p^+=\begin{bmatrix}
		p_1 & 0   & 0   \\
		0   & p_2 & 0   \\
		0   & 0   & p_3
	\end{bmatrix}
\end{equation}

\begin{equation}
	p^\wedge=\begin{bmatrix}
		0    & -p_3 & p_2  \\
		p_3  & 0    & -p_1 \\
		-p_2 & p_1  & 0
	\end{bmatrix}
\end{equation}

\begin{equation}
	\xi^\wedge
	=\begin{bmatrix}
		\phi \\ \rho
	\end{bmatrix}^\wedge
	=\begin{bmatrix}
		\phi^\wedge & \rho \\
		0           & 0
	\end{bmatrix}
\end{equation}

\begin{equation}
	p^\odot=\begin{bmatrix}
		-p^\wedge & I_{3\times3}
	\end{bmatrix}
\end{equation}

\par 可以注意到,$\wedge$符号使用了两次,一个作用于三维向量得到反对称矩阵,一个作用于李代数扰动增量得到雅克比矩阵,$\odot$表示先旋转后平移。
\par 此外,下面的推导中隐含有齐次坐标和非齐次坐标之间的转换,为了简便起见,不一一说明。

\subsection{LaserOdometry}
已知相邻两帧点云$P_1$和$P_2$之间的初始位姿,通过"点-直线"和"点-面"距离最小化进行位姿优化。
\subsubsection{点-直线距离最小化}
已知初始位姿的李代数为$\xi$,对于$\forall p \in P_1$,有对应点$p_1,p_2 \in P_2$构成直线,使得点$p$到直线$p_1p_2$的距离最短,该距离用向量表示如下
\begin{equation}
	d_{\epsilon} =
	\frac{|\vecba{p p_1} \times \vecba{p p_2}|}{|\vecba{p_1 p_2}|}
\end{equation}
用李代数表示如下
\begin{equation}
	\vecba{d_{\epsilon}} =
	\frac{(p_1-e(\xi ^\wedge)p)^\wedge(p_2-e(\xi ^\wedge)p)}{|\vecba{p_1 p_2}|}
\end{equation}
\par 记$p'=e(\xi ^\wedge)p$,$f=(p_1-e(\xi ^\wedge)p) ^\wedge (p_2-e(\xi ^\wedge)p)$,使用李代数扰动模型求导,可以得到
\begin{equation}
	\begin{split}
		\frac{\partial f}{\partial \delta \xi}
		&=
		\lim\limits_{\delta \xi \rightarrow 0}
		\frac{(p_1-e(\delta \xi ^\wedge) p')
			^\wedge
			(p_2-e(\delta \xi ^\wedge) p')
			-
			(p_1-p') ^\wedge (p_2-p')
		}{\delta \xi} \\
		&=
		\lim\limits_{\delta \xi \rightarrow 0}
		\frac{(p_1-(1+\delta \xi ^\wedge) p')
			^\wedge
			(p_2-(1+\delta \xi ^\wedge) p')
			-
			(p_1-p') ^\wedge (p_2-p')
		}{\delta \xi} \\
		&=
		\lim\limits_{\delta \xi \rightarrow 0}
		\frac{-(p_1-p')^\wedge\delta\xi^\wedge p'- (\delta\xi^\wedge p')^\wedge(p_2-p')}{\delta \xi} \\
		&=
		- (p_1-p')^\wedge p'^\odot + (p_2-p')^\wedge p'^\odot \\
		&=
		(p_2 - p_1)^\wedge p'^\odot
	\end{split}
\end{equation}

\subsubsection{点-面距离最小化}
已知初始位姿的李代数为$\xi$,对于$\forall p \in P_1$,有对应点$p_1,p_2,p_3 \in P_2$构成平面,使得点$p$到平面$p_1p_2p_3$的距离最短,该距离用向量表示如下
\begin{equation}
	d_{\epsilon} =
	\frac{|(\vecba{p p_1} \times \vecba{p p_2}) \cdot \vecba{p p_3}|}{2|S_{p_1 p_2 p_3}|}
\end{equation}
用李代数表示如下
\begin{equation}
	\vecba{d_{\epsilon}} =
	\frac{(p_3-e(\xi ^\wedge)p)^{+}((p_1-e(\xi ^\wedge)p)^\wedge(p_2-e(\xi ^\wedge)p))}{2|S_{p_1 p_2 p_3}|}
\end{equation}
\par 记$p'=e(\xi ^\wedge)p$,$p'_1=p_1-p'$,$p'_2=p_2-p'$,$p'_3=p_3-p'$,$h=(p_3-e(\xi ^\wedge)p)
	^{+}(
	(p_1-e(\xi ^\wedge)p)
	^\wedge
	(p_2-e(\xi ^\wedge)p))$,使用李代数扰动模型求导,可以得到
\begin{equation}
	\begin{split}
		\frac{\partial h}{\partial \delta \xi}
		&=
		{p'}_3^{+} {p'}_2^\wedge p'^\odot -
		{p'}_3^{+} {p'}_1^\wedge p'^\odot -
		({p'}_1^\wedge p'_2)^{+} p'^\odot \\
		&=
		{p'}_3^{+} ({p'}_2-{p'}_1)^\wedge p'^\odot + ({p'}_2^\wedge p'_1)^{+} p'^\odot
	\end{split}
\end{equation}

\subsubsection{实现}
上面的公式来自于loam论文,推导结果是我自己得来的,相较于loam代码中给出的偏导形式要简单得多,但仍较为复杂。实际上,实现时可以使用较为简单的公式。
\par "点-线"可以使用下面的公式
\begin{equation}
	\vecba{d_{\epsilon}} =
	\frac{(p_2-p_1)^\wedge(e(\xi ^\wedge)p-p_1)}{|\vecba{p_1 p_2}|}
\end{equation}
\begin{equation}
	\frac{\partial \vecba{d_{\epsilon}}}{\partial\delta\xi}
	=
	\frac{(p_2-p_1)^\wedge(e(\xi^\wedge)p)^\odot}{|\vecba{p_2p_2}|}
\end{equation}

\par "点-面"可以使用下面的公式
\begin{equation}
	\vecba{d_{\epsilon}} =
	\frac{(p_3-p_1)^{+}(p_2-p_1)^\wedge(e(\xi ^\wedge)p-p_1)}{2|S_{p_1 p_2 p_3}|}
\end{equation}
\begin{equation}
	\frac{\partial \vecba{d_{\epsilon}}}{\partial\delta\xi}
	=
	\frac{(p_3-p_1)^{+}(p_2-p_1)^\wedge(e(\xi ^\wedge)p)^\odot}{2|S_{p_1 p_2 p_3}|}
\end{equation}

\par MATLAB代码推导:
\small
\begin{lstlisting}{language=Matlab}
	% matrix prefix:
	% D means 'diag', S means 'skew', d means 'delta'
	%
	% variable name:
	% x0 := p0_x
	% z10 := p0_z - p1_z
	
	%% LaserOdometry derivation
	syms x0 y0 z0 real
	syms x10 y10 z10 real
	syms x12 y12 z12 real
	syms x13 y13 z13 real
	Dd13 = diag([x13 y13 z13]);
	Sd12 = [0 -z12 y12; z12 0 -x12; -y12 x12 0];
	S0 = [0 -z0 y0; z0 0 -x0; -y0 x0 0];
	d10 = [x10 y10 z10]';
	% use 2 points to represent a line
	disp('point-line')
	J = Sd12 * [-S0 eye(3)]
	r = Sd12 * d10
	% use 3 points to represent a plane
	disp('point-plane')
	J = Dd13 * J
	r = Dd13 * r
	
	%% LaserMapping derivation
	syms x1 y1 z1
	syms vx vy vz
	Sv = [0 -vz vy; vz 0 -vx; -vy vx 0];
	Dv = diag([vx vy vz]);
	% use a point and a direction vector to represent a line
	disp('point-line')
	J = Sv * [-S0 eye(3)]
	r = Sv * d10
	% use a point and a normal vector to represent a plane
	disp('point-plane')
	J = Dv * [-S0 eye(3)]
	r = Dv * d10
	\end{lstlisting}

\subsubsection{注意}
上面公式中的$\vecba{d_\epsilon}$表示误差向量,${\partial \vecba{d_{\epsilon}}}/{\partial\delta\xi}$表示误差向量相对于李代数增量$\delta\xi$的偏导(雅克比矩阵)。利用误差项和雅克比矩阵,使用LM方法,可以迭代得出位姿估计值。

\subsection{LaserMapping}
LaserOdometry对相邻两帧之间进行匹配,由于累计误差的存在和里程计本身的不可靠性,loam通过LaserMapping进行位姿的优化,其原理是,将新的扫描帧与附近的点云通过匹配确定位姿,同时在搜寻”点-线“和”点-面“时进行了细致的处理(过程为搜索线时使用特征值分解,看是否存在特征值较大的主方向;搜索面时使用最小二乘计算平面,若有点距离平面较远,则不是合法的”点-面“对)。
\par 优化问题的计算与上面的完全相同。
\par 已知优化后点$p'=e(\xi ^\wedge)p$,对应直线的经过点$p_0$,直线方向用\textbf{单位向量}$v$表示,则"点-线"可以使用下面的公式
\begin{equation}
	\vecba{d_{\epsilon}} = v^\wedge(e(\xi ^\wedge)p-p_0)
\end{equation}
\begin{equation}
	\frac{\partial \vecba{d_{\epsilon}}}{\partial\delta\xi}
	=
	v^\wedge(e(\xi^\wedge)p)^\odot
\end{equation}
\par 已知优化后点$p'=e(\xi ^\wedge)p$,对应平面通过点$p_0$,平面法线方向用\textbf{单位向量}$v$表示,则"点-面"可以使用下面的公式
\begin{equation}
	\vecba{d_{\epsilon}} = v^{+}(e(\xi ^\wedge)p-p_0)
\end{equation}
\begin{equation}
	\frac{\partial \vecba{d_{\epsilon}}}{\partial\delta\xi}
	=
	v^{+}(e(\xi^\wedge)p)^\odot
\end{equation}
