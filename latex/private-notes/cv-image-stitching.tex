\section{全景相片拼接}
记像素点坐标为p,校正后像片内参为K,像空间辅助坐标系和世界坐标系间的变换为R和t。
\begin{equation}
  sp=K(RW+t)
\end{equation}

把$K^{-1}$移到左边,得到
\begin{equation}
  sK^{-1}p=RW+t
\end{equation}

令$P=K^{-1}p$,则
\begin{equation}
  sP=RW+t \label{camera_base}
\end{equation}

\subsection{推导一}
由$P$的最后一行为1可知,
\begin{equation}
  s=R^{(3)}W+t^{(3)}
\end{equation}

代入公式(\ref{camera_base})可知
\begin{equation}
  (PR^{(3)}-R)W=t-Pt^{(3)} \label{wrong_label}
\end{equation}

对于左右两个相机,有
\begin{equation}
  (P_1 R_1^{(3)} - R_1)(t_2 - P_2 t_2^{(3)})=
  (P_2 R_2^{(3)} - R_2)(t_1 - P_1 t_1^{(3)})
\end{equation}

该公式中,除了$P_2$是待求量,其它全都是已知的。
不妨记$k1=R_2^{(3)} (t_1 - P_1 t_1^{(3)})$,
$C=P_1 R_1^{(3)} - R_1$,
$k2=t_2^{(3)}$,
$V=(P_1 R_1^{(3)} - R_1)t_2 + R_2(t_1 - P_1 t_1^{(3)})$,则上式可化简为
\begin{equation}
  P_2 k_1 + C P_2 k_2 = V
\end{equation}

考虑到$k_1$ $k_2$都是1x1方阵,可以看成实数,公式可以进一步化简为
\begin{equation}
  (k_1 E + k_2 C ) P_2 = V
\end{equation}

求解上式可得$P_2$坐标。

很遗憾地发现,上述推导从公式(\ref{wrong_label})开始就是错的,W无法被消去。

\subsection{推导二}
本推导的目的是通过公式(\ref{camera_base})将像片上的像点坐标投影到虚拟像空间坐标系,从而进一步投影到圆柱面或球面,生成全景影像。
设虚拟像空间坐标系表示为$R',t'$,相片的像空间辅助坐标系表示为$R, t$,则由公式(\ref{camera_base})可以得到
\begin{equation}
  W = R^{-1} (P - t) = R'^{-1} (P' - t') \label{quanjingpinjie}
\end{equation}
上式中的$P$对应于于公式(\ref{camera_base})中的$sP$,是像空间辅助坐标系下的坐标。
因此有
\begin{equation}
  P = s K^{-1} p \label{image_to_xyz}
\end{equation}

将公式(\ref{image_to_xyz})代入公式(\ref{quanjingpinjie}),得到
\begin{equation}
  P' = R' R^{-1} (s K^{-1} p - t) + t'
\end{equation}

记$\vec{v_1} = R' R^{-1} K^{-1} p = \begin{bmatrix}
    a & b & c
  \end{bmatrix}' $,$\vec{v_2} = t' - R' R^{-1} t = \begin{bmatrix}
    m & n & o
  \end{bmatrix}'$,则最终可以得到公式
\begin{equation} \label{get_P_}
  P' = s
  \begin{bmatrix}
    a \\ b \\ c
  \end{bmatrix}
  +
  \begin{bmatrix}
    m \\ n \\ o
  \end{bmatrix}
\end{equation}

在圆柱面投影中,结合公式${P'}_x^2 + {P'}_y^2 = R^2$可以得到一个一元二次方程。该方程必然有正负两个解,负解舍去,正解为
\begin{equation}
  \begin{split}
    s
    &= \frac{\sqrt{(\vec{v_1}\cdot\vec{v_2})^2 - \vec{v_1}^2(\vec{v_2}^2-R^2)} - \vec{v_1}\cdot\vec{v_2}}{\vec{v_1}^2} \\
    &= \frac{\sqrt{(am+bn)^2-(a^2+b^2)(m^2+n^2-R^2)}-(am+bn)}{a^2+b^2} \\
    &= \frac{\sqrt{(a^2+b^2)R^2 - (an-bm)^2}-(am+bn)}{a^2+b^2}
  \end{split}
\end{equation}

在球面投影中,结合公式式${P'}_x^2 + {P'}_y^2 + {P'}_z^2 = R^2$可以得到一个一元二次方程。该方程必然有正负两个解,负解舍去,正解为
\begin{equation}
  \begin{split}
    s
    &= \frac{\sqrt{(\vec{v_1}\cdot\vec{v_2})^2 - \vec{v_1}^2(\vec{v_2}^2-R^2)} - \vec{v_1}\cdot\vec{v_2}}{\vec{v_1}^2} \\
    &=\frac{\sqrt{(a m+b n+c o)^2-(a^2+b^2+c^2) (m^2+n^2+o^2-R^2)}-(am+bn+co)}{a^2+b^2+c^2}
  \end{split}
\end{equation}

\subsection{误差来源}
\subsubsection{投影差}
从上面的推导可以知道,虚拟坐标系原点和像素对应的空间点的距离被假定成了R,这会导致距离并非R的像素点在每张图片上都有一个投影偏移,这种偏移在拼接中更为致命:若真实距离大于R,则两个相片拼接重合的部分将偏多,否则偏少。
\par
事实上,鉴于R都是假定的,我们甚至可以假设$P_x^2 + P_y^2 = R^2$,这样我们可以直接算出s的值,进一步通过公式(\ref{get_P_}),得到P'点的坐标。

\subsection{使用深度图进行投影}
\par 相机成像模型
\begin{equation}
  s p = K (R W + t) , s>0
\end{equation}
\par 深度=物点到相机中心距离
\begin{equation}
  depth(p) = |W + R^{-1} t|
\end{equation}

\par 由成像模型得到
\begin{equation}
  s R^{-1} K^{-1} p = W + R^{-1} t
\end{equation}
\par 结合深度公式得到
\begin{equation}
  s | R^{-1} K^{-1} p | = depth(p), s>0
\end{equation}
\par 因此,有
\begin{equation}
  s  = \frac{depth(p)}{| R^{-1} K^{-1} p |}
\end{equation}
\par 进而计算出$W$,再结合
\begin{equation}
  P' = R' W + t'
\end{equation}
\par 计算出$P'$即可。
\par 当前方法存在的问题:
1. 本方法是经典的几何方法,不考虑现实情况数学上是严密的,然而真实的相机内参外参、遮挡等都会影响结果,因此拼接结果不会完美;
2. 本方法要注意的是:深度图必须要是平滑的,遮挡的要特殊考虑;在非平滑环境中效果会很差。

\subsection{使用深度图进行投影 fucking final}
\par 相机成像模型
\begin{equation}
  s p = K (R W + t) , s>0
\end{equation}
\par 深度 = s
\begin{equation}
  depth(p) = s
\end{equation}

\par 由上述公式得
\begin{equation}
  W = R^{-1} ( s K^{-1} p - t)
\end{equation}
\par 再结合
\begin{equation}
  P' = R' W + t'
\end{equation}
\par 计算出$P'$,然后就可以给出点在全景图上的像素了。
\par TIPS:
\par 逆映射指的是已知全景图上的像素和相机的深度图,给出对应相机的图像像素,很显然逆映射无法用经典的代数表达式来表示。
\par 然而如果已知确定形式的$f((x,y))=(x',y')$的映射,可以内插出$g((x',y'))=(x,y)$。
