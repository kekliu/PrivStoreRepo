\section{SLAM主流算法}
\subsection{匹配方法}
\subsubsection{特征点/线/面匹配(激光和视觉)}
\paragraph{ICP}
\paragraph{loam}
\subsubsection{直接法(视觉)}
\paragraph{光流法}
\paragraph{直接法}
\subsubsection{概率格网(激光)}
\subsection{滤波器方法}
\subsubsection{贝叶斯定理}
\par 约定以下记号:$x_k$是系统状态量(如位置、速度、航向),$u_k$是控制量(如角速度、力),通过贝叶斯定理可以得到:
\begin{equation}
	\begin{split}
		p(x_k|x_{k-1},u_k,z_k) & = \frac{p(x_k,x_{k-1},u_k,z_k)}{p(x_{k-1},u_k,z_k)} \\
		& = \frac{p(z_k|x_k,x_{k-1},u_k)p(x_k,x_{k-1},u_k)}{p(z_k|x_{k-1},u_k)p(x_{k-1},u_k)} \\
		&= \frac{p(z_k|x_k,x_{k-1},u_k)p(x_k|x_{k-1},u_k)p(x_{k-1},u_k)}{p(z_k|x_{k-1},u_k)p(x_{k-1},u_k)} \\
		&= \frac{p(z_k|x_k,x_{k-1},u_k)p(x_k|x_{k-1},u_k)}{p(z_k|x_{k-1},u_k)}
	\end{split}
\end{equation}
\par 在滤波器的应用场景中,机器人的运动状态满足马尔可夫假设,即当前状态只与上一刻状态相关,与更早的状态无关。由马尔可夫假设得到$p(z_k|x_k,x_{k-1},u_k)=p(z_k|x_k)$和$p(z_k|x_{k-1},u_k)=p(z_k)$,代入上述公式得:
\begin{equation}
	\begin{split}
		p(x_k|x_{k-1},u_k,z_k) & = \frac{p(z_k|x_k)p(x_k|x_{k-1},u_k)}{p(z_k)}
	\end{split}
\end{equation}
\par 其中,$p(z_k|x_k)$是似然,$p(x_k|x_{k-1},u_k)$是先验。
滤波器方法即是先描述先验和似然的数学模型,再取状态值的最大后验的一种方法。


\subsubsection{卡尔曼滤波/线性高斯滤波: KF}
已知状态方程和观测方程:
\begin{equation}
	\left\{
	\begin{array}{lr}
		x_k = F x_{k-1} + B u_k + w \\
		z_k = H x_k + v
	\end{array}
	\right.
\end{equation}
\par 其中,$x_k$是系统状态量(如位置、速度、航向),$u_k$是控制量(如角速度、力),$F$是状态转移矩阵,$B$是输入控制矩阵,$w$是过程噪声,$z_k$是量测量,$H$是状态向量到量测量的变换矩阵,$v$是量测噪声。请注意,上面的$F,B,w,H,v$都有下标t,这里为了方便书写略去不写。过程噪声$w$和量测噪声$v$满足零均值高斯分布:
\begin{equation}
	w \sim N(0,R),v \sim N(0,Q)
\end{equation}
\par 进一步得到下面的方程组:
\begin{equation}
	\left\{
	\begin{split}
		p(x_k|x_{k-1},u_k)
		&=N(\bar{x}_k,\bar{P}_k)
		=N(F \hat{x}_{k-1} + B u_k, F \hat{P}_{k-1} F^T + R) \\
		p(z_k|x_k) &=N(H x_k,Q) \\
		p(x_k|z_k) &= \frac{p(z_k|x_k)p(x_k|x_{k-1},u_k)}{p(z_k)} \propto p(z_k|x_k)p(x_k|x_{k-1},u_k)
	\end{split}
	\right.
\end{equation}
\par 具体来说:
\begin{equation}
	N(\hat{x}_k,\hat{P}_k) \propto N(\bar{x}_k,\bar{P}_k) \cdot N(H x_k,Q)
\end{equation}
\par 考虑等式两边的指数部分,可以得到
\begin{equation}
	(x_k-\hat{x}_k)^T\hat{P}_k^{-1}(x_k-\hat{x}_k)
	=
	(z_k-H x_k)^T Q^{-1} (z_k-H x_k)
	+
	(x_k-\hat{x}_k)^T \bar{P}_k^{-1} (x_k-\hat{x}_k)
\end{equation}
\par 令等号两侧$x_k$的一次项和二次项系数相等,可以得到卡尔曼滤波器的结果:
\par a. 预测
\begin{equation}
	\begin{split}
		\bar{x}_k&=F\hat{x}_{k-1} + B u_k \\
		\bar{P}_k&=F\hat{P}_{k-1}F^T+R
	\end{split}
\end{equation}
\par b. 更新
\begin{equation}
	\begin{split}
		K&=\bar{P}_k H^T(H \bar{P}_k H^T + Q)^{-1} \\
		\hat{x}_k&=\bar{t}_k+K(z_k-H \bar{x}_k) \\
		\hat{P}_k&=(I-K H)\bar{P}_k
	\end{split}
\end{equation}


\subsubsection{扩展卡尔曼滤波: EKF}
\subsubsection{无迹卡尔曼滤波: UKF}
\subsubsection{粒子滤波: PF}

\subsection{图优化}
\subsubsection{关键帧BA}
