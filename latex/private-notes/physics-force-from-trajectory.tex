\section{已知运动轨迹,推算向心力}
\subsection{问题}
已知物体仅在向心力的作用下沿着给定轨迹运动,求该向心力的规律。
\subsection{推导}
为了方便推导,我们假设向心力来自原点。设某个时刻,物体的位置矢量为 $ \vec{r} =(x,y) $,速度矢量为 $ \vec{v}=(v_x,v_y)=v_x (1,y') $,其中,$ y'=\frac{dy}{dx} $。
\\
由动量守恒知,$ \vec{r}\times\vec{v}=const $,则$ v_x\propto\frac{1}{y-y'x} $,令
\begin{equation} v_x=\frac{k}{y-y'x} \end{equation}
则加速度
\begin{equation}
  \begin{split}
    \vec{a} &= \frac{d\vec{v}}{dt} = \frac{d\vec{v}}{dx} \frac{dx}{dt} = \frac{d\vec{v}}{dx} v_x \\
    &= \frac{k^2 y''}{(y-y'x)^3} (x,y)
  \end{split}
\end{equation}
\subsection{推论}
\noindent
通过该公式可以得到一些有趣的结果:
\\若轨迹为圆锥曲线,则指向曲线焦点的向心力反比于距离的平方;
\\若轨迹为椭圆轨道,则指向椭圆中心的向心力正比于距离;
\\若轨迹为任意曲线,则仍可通过上述公式计算向心力。
\subsection{求指向无穷远处的力的规律}
给定一个曲线,假设其只受到来自无穷远方向的力,求该力的规律。
取力的方向为y轴,与y轴垂直的为x轴。
设某个时刻,物体的位置矢量为 $ \vec{r} =(x,y) $。则由水平方向不受力
易知 $dx / v_x = dt$ ,竖直方向有
\begin{equation}
  v_y = v_x y'
\end{equation}
\begin{equation}
  F = m a_y = m v_x \frac{y'}{dt} = m v_x \frac{dy'}{dx / v_x} = m v_x^2 y''
\end{equation}
\paragraph{结论}
指向无穷远处的力与二阶导数成正比。
特别地,抛物线受到的力为常量。
