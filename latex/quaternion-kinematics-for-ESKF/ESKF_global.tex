% !TEX root = kinematics.tex


%\clearpage
%%%%%%%%%%%%%%%%%%%%%%%%%%%%%%%%%%%%%%%%%%%%%%%%%%%%%%%%%%%%%%
\section{The ESKF using global angular errors}
\label{sec:ESKFglobal}

We explore in this section the implications of having the angular error defined in the global reference, as opposed to the local definition we have used so far. 
We retrace the development of sections \ref{sec:es-kinematics} and \ref{sec:fusion}, and particularize the subsections that present changes \wrt the new definition.

A global definition of the angular error $\delta\bftheta$ implies a composition \emph{on the left hand side}, \ie,
%
\begin{equation*}
\bfq_t = \delta\bfq\ \ot \bfq = \bfq\{\delta\bftheta\} \ot \bfq~.
\end{equation*}

We remark for the sake of completeness that we keep the local definition of the angular rates vector $\bfomega$, \ie, $\dot\bfq=\frac12\bfq\ot\bfomega$ in continuous time, and therefore $\bfq \gets \bfq\ot\bfq\{\bfomega\Dt\}$ in discrete time, regardless of the angular error being defined globally. 
This is so for convenience, as the measure of the angular rates provided by the gyrometers is in body frame, that is, local.

%=============================================================
\subsection{System kinematics in continuous time}

%-------------------------------------------------------------
\subsubsection{The true- and nominal-state kinematics}

True and nominal kinematics do not involve errors and their equations are unchanged.

%-------------------------------------------------------------
\subsubsection{The error-state kinematics}

We start by writing the equations of the error-state kinematics, and proceed afterwards with comments and proofs.
%
\begin{subequations}\label{equ:efullglobal}
%
\begin{align}
\dot{\delta\bfp} &= \delta\bfv \label{equ:epglobal} \\
\dot{\delta\bfv} &= -\hatx{\bfR(\bfa_m-b\bfa_b)}\delta\bftheta - \bfR\delta\bfa_b + \delta\bfg - \bfR\bfa_n \label{equ:evglobal}\\
\dot{\delta\bftheta} &= -\bfR\delta\bfomega_b - \bfR\bfomega_n \label{equ:etglobal}\\
\dot{\delta\bfa_b} &= \bfa_w \label{equ:eabglobal}\\
\dot{\delta\bfomega_b} &= \bfomega_w \label{equ:ewbglobal}\\
\dot{\delta\bfg} &= 0 ~, \label{equ:egglobal}
\end{align}%
\end{subequations}%
%
where, again, all equations except those of $\dot{\delta\bfv}$ and $\dot{\delta\bftheta}$ are trivial. 
The non-trivial expressions are developed below.


\paragraph{Equation \eqRef{equ:evglobal}: The linear velocity error.}

We wish to determine $\dot{\delta\bfv}$, the dynamics of the velocity errors. 
We start with the following relations
%
%
\begin{align}
\bfR_t &= (\bfI+\hatx{\delta\bftheta})\bfR  + O(\norm{\delta\bftheta}^2) \label{equ:Rtglobal}\\
\dot\bfv &= \bfR\bfa_\cB + \bfg~, \label{equ:vdot2global}
\end{align}%
%
where \eqRef{equ:Rtglobal} is the small-signal approximation of $\bfR_t$ using a globally defined error, and in \eqRef{equ:vdot2global} we introduced $\bfa_\cB$ and $\delta\bfa_\cB$ as the large- and small- signal accelerations in body frame, defined in\eqRef{equ:nomacc} and \eqRef{equ:pertacc}, as we did for the locally-defined case.

We proceed by writing the expression \eqRef{equ:vel} of $\dot\bfv_t$ in two different forms (left and right developments), where the terms $O(\norm{\delta\bftheta}^2)$ have been ignored,
%
%
\begin{align*}
\dot\bfv+\dot{\delta\bfv} =& \eqbox{\dot\bfv_t} = (\bfI + \hatx{\delta\bftheta})\bfR(\bfa_\cB+\delta\bfa_\cB) + \bfg + \delta\bfg \\
\bfR\bfa_\cB+\bfg+\dot{\delta\bfv} =&
~~~~~~= \bfR\bfa_\cB+\bfR\delta\bfa_\cB + \hatx{\delta\bftheta}\bfR\bfa_\cB + \hatx{\delta\bftheta}\bfR\delta\bfa_\cB + \bfg + \delta\bfg 
\end{align*}%
%
This leads after removing $\bfR\bfa_\cB+\bfg$ from left and right to
%
\begin{equation}
\dot{\delta\bfv} = \bfR\delta\bfa_\cB + \hatx{\delta\bftheta}\bfR(\bfa_\cB + \delta\bfa_\cB) + \delta\bfg
\end{equation}%
%
Eliminating the second order terms and reorganizing some cross-products (with $\hatx{\bfa}\bfb=-\hatx{\bfb}\bfa$), we get
%
\begin{equation}
\dot{\delta\bfv} = \bfR\delta\bfa_\cB - \hatx{\bfR\bfa_\cB}\delta\bftheta + \delta\bfg~,
\end{equation}%
%
and finally, recalling \eqRef{equ:nomacc} and \eqRef{equ:pertacc} and rearranging, we obtain  the expression of the derivative of the velocity error,
%
\begin{equation}
\eqbox{{\dot{\delta\bfv} = -\hatx{\bfR(\bfa_m-\bfa_b)}\delta\bftheta - \bfR\delta\bfa_b + \delta\bfg - \bfR\bfa_n }}
\end{equation}%


\paragraph{Equation \eqRef{equ:etglobal}: The orientation error.}

We start by writing the true- and nominal- definitions of the quaternion derivatives,
%
%
\begin{align}
\dot{\bfq_t} &= \frac{1}{2}\bfq_t\ot\bfomega_t \\
\dot{\bfq} &= \frac{1}{2}\bfq\ot\bfomega~,
\end{align}%
%
and remind that we are using a globally-defined angular error, \ie,
%
\begin{equation}
\bfq_t = \delta\bfq\ot\bfq~.
\end{equation}
%
As we did for the locally-defined error case, we also group large- and small-signal angular rates \eqsRef{equ:nomangrate}{equ:pertangrate}. 
We proceed by computing $\dot{\bfq_t}$ by two different means (left and right developments),
%
%
\begin{align*}
\dot{{({\delta\bfq}\ot\bfq)}} =& \eqbox{\dot{\bfq_t}} = \frac{1}{2}\bfq_t\ot\bfomega_t \\
\dot{\delta\bfq}\ot{\bfq} + \delta\bfq\ot\dot{{\bfq}} =&~~~~~~= \frac{1}{2}\delta\bfq\ot\bfq\ot\bfomega_t \\
\dot{\delta\bfq}\ot{\bfq} + \frac{1}{2}\delta\bfq\ot\bfq\ot\bfomega =&&  
\end{align*}%
%
Having $\bfomega_t = \bfomega + \delta\bfomega$, this reduces to
%
\begin{equation}
\dot{\delta\bfq}\ot{\bfq} = \frac{1}{2}\delta\bfq\ot\bfq\ot\delta\bfomega~.
\end{equation}
%
Right-multiplying left and right terms by $\bfq^*$, and recalling that $\bfq\ot\delta\bfomega\ot\bfq^* \equiv \bfR\delta\bfomega$, we can further develop as follows,
%
%
\begin{align}
\dot{\delta\bfq} &= \frac{1}{2}\delta\bfq\ot\bfq\ot\delta\bfomega\ot\bfq^* \nonumber \\
&= \frac{1}{2}\delta\bfq\ot(\bfR\delta\bfomega) \nonumber \\
&= \frac{1}{2}\delta\bfq\ot\delta\bfomega_G~,
\end{align}%
%
with $\delta\bfomega_G \triangleq \bfR\delta\bfomega$ the small-signal angular rate expressed in the global frame. Then,
%
%
\begin{align}
\begin{bmatrix}
0\\\dot{\delta\bftheta}
\end{bmatrix} = \eqbox{2\dot{{\delta\bfq}}} 
&= {\delta\bfq}\ot\delta\bfomega_G \nonumber \\
&= \Omega(\delta\bfomega_G)\,\delta\bfq \nonumber \\
&= \begin{bmatrix}
0 & -\delta\bfomega_G\tr \\
\delta\bfomega_G & -\hatx{\delta\bfomega_G} 
\end{bmatrix}\begin{bmatrix}
1 \\
\delta\bftheta/2
\end{bmatrix} + O(\norm{\delta\bftheta}^2) ~,
\end{align}%
%
which results in one scalar- and one vector- equalities
%
\begin{subequations}
%
\begin{align}
0 &= \delta\bfomega_G\tr\delta\bftheta + O(|\delta\bftheta|^2) \\
\dot{\delta\bftheta} &= \delta\bfomega_G - \frac{1}{2}\hatx{\delta\bfomega_G}\delta\bftheta + O(\norm{\delta\bftheta}^2).
\end{align}%
\end{subequations}
%
The first equation leads to $\delta\bfomega_G\tr\delta\bftheta = O(\norm{\delta\bftheta}^2)$, which is formed by second-order infinitesimals, not very useful. 
The second equation yields, after neglecting all second-order terms,
%
\begin{equation}
\dot{\delta\bftheta} = \delta\bfomega_G = \bfR\delta\bfomega ~.
\end{equation}%
%
Finally, recalling %\eqRef{equ:nomangrate} and 
\eqRef{equ:pertangrate}, we obtain the linearized dynamics of the global angular error,
%
\begin{equation}
\eqbox{\dot{\delta\bftheta} = - \bfR\delta\bfomega_b - \bfR\bfomega_n}\ .
\end{equation}%

%=============================================================
\subsection{System kinematics in discrete time}
%-------------------------------------------------------------
\subsubsection{The nominal state}
The nominal state equations do not involve errors and are therefore the same as in the case where the orientation error is defined locally. 

%-------------------------------------------------------------
\subsubsection{The error state}

Using Euler integration, we obtain the following set of differences equations,
%
\begin{subequations}
%
\begin{align}
\delta\bfp &\gets \delta\bfp + \delta\bfv\,\Dt \\
\delta\bfv &\gets \delta\bfv + (-\hatx{\bfR(\bfa_{m}-\bfa_{b})}\delta\bftheta - \bfR\delta\bfa_{b} + \delta\bfg)\Dt + \bfv_\bfi \\
\delta\bftheta &\gets \delta\bftheta - \bfR\delta\bfomega_{b} \Dt + \bftheta_\bfi \\
\delta\bfa_{b} &\gets  \delta\bfa_{b} + \bfa_\bfi \\
\delta\bfomega_{b} &\gets \delta\bfomega_{b} + \bfomega_\bfi \\
\delta\bfg &\gets \delta\bfg .
\end{align}%
\end{subequations}

%-------------------------------------------------------------
\subsubsection{The error state Jacobian and perturbation matrices}

The Transition matrix is obtained by simple inspection of the equations above,
%
\begin{equation}
\bfF_\bfx = %\pjac{f}{\delta\bfx}{\bfx,\bfu_m} = 
\begin{bmatrix}
\bfI & \bfI\Dt & 0                             & 0               & 0                     & 0 \\
0 & \bfI    & \eqbox{-\hatx{\bfR(\bfa_m-\bfa_b)}\Dt}     & -\bfR\Dt            & 0                     & \bfI\Dt \\
0 & 0    & \eqbox{\bfI}   & 0               & \eqbox{-\bfR\Dt}                  & 0 \\
0 & 0    & 0                             & \bfI & 0                     & 0 \\
0 & 0    & 0                             & 0               & \bfI  & 0 \\
0 & 0    & 0                             & 0               & 0                     & \bfI \\
\end{bmatrix}~.
\end{equation}

We observe three changes \wrt the case with a locally-defined angular error (compare the boxed terms in the Jacobian above to the ones in \eqRef{equ:Fx_local_euler}); these changes are summarized in \tabRef{tab:local_to_global}.


The perturbation Jacobian and the perturbation matrix are unchanged after considering isotropic noises and the developments of \appRef{sec:IntNoise},
%
\begin{equation}
\bfF_\bfi = %\pjac{f}{\bfi}{\bfx,\bfu_m} = 
\begin{bmatrix}
0 & 0 & 0 & 0 \\
\bfI & 0 & 0 & 0 \\
0 & \bfI & 0 & 0 \\
0 & 0 & \bfI & 0 \\
0 & 0 & 0 & \bfI \\
0 & 0 & 0 & 0 
\end{bmatrix}  
\quad,\quad
\bfQ_\bfi = \begin{bmatrix}
\bfV_\bfi & 0        & 0      & 0 \\ 
0      & \bfTheta_\bfi & 0      & 0 \\ 
0      & 0        & \bfA_\bfi & 0 \\ 
0      & 0        & 0      & \bfOmega_\bfi 
\end{bmatrix}~.
\end{equation}%
%


%=============================================================
\subsection{Fusing with complementary sensory data}

The fusing equations involving the ESKF machinery vary only slightly when considering global angular errors. 
We revise these variations in the error state observation via ESKF correction, the injection of the error into the nominal state, and the reset step.

%-------------------------------------------------------------
\subsubsection{Error state observation}

The only difference \wrt the local error definition is in the Jacobian block of the observation function that relates the orientation to the angular error. 
This new block is developed below.

Using \eqsRef{equ:quatMatProd}{equ:quatMatrix} and the first-order approximation $\delta\bfq\rightarrow \begin{bmatrix}
1 \\ \frac12\delta\bftheta
\end{bmatrix}$, the quaternion term $\bfQ_{\delta\bftheta}$ may be derived as follows,
%
\begin{subequations}
\begin{align}
\bfQ_{\delta\bftheta} \triangleq \pjac{(\delta\bfq\otimes\bfq)}{\delta\bftheta}{\bfq} 
&=\pjac{(\delta\bfq\otimes\bfq)}{\delta\bfq}{\bfq} \pjac{\delta\bfq}{\delta\bftheta}{\hat{\delta\bftheta}=0} \\
&= [\bfq]_R\,\frac12\begin{bmatrix}
0 & 0 & 0 \\
1 & 0 & 0 \\
0 & 1 & 0 \\
0 & 0 & 1 \\
\end{bmatrix} \\
&= \frac{1}{2}\begin{bmatrix}
-q_x &-q_y &-q_z\\
 q_w & q_z &-q_y\\
-q_z & q_w & q_x\\
 q_y &-q_x & q_w\\
\end{bmatrix}~.
\end{align}
\end{subequations}


%-------------------------------------------------------------
\subsubsection{Injection of the observed error into the nominal state}

The composition $\bfx \gets \bfx\oplus\hat{\delta\bfx}$ of the nominal and error states is depicted as follows,
%
\begin{subequations}
%
\begin{align}
\bfp &\gets \bfp + \delta\bfp \\
\bfv &\gets \bfv + \delta\bfv \\
\bfq &\gets \bfq\{\hat{\delta\bftheta}\} \ot \bfq \\ 
\bfa_b &\gets \bfa_b + \delta\bfa_b \\
\bfomega_b &\gets \bfomega_b + \delta\bfomega_b \\
\bfg &\gets \bfg + \delta\bfg ~.
\end{align}%
\end{subequations}
%
where only the equation for the quaternion update has been affected. 
This is summarized in \tabRef{tab:local_to_global}.

%-------------------------------------------------------------
\subsubsection{ESKF reset}

The ESKF error mean is reset, and the covariance updated, according to,
%
%
\begin{align}
\hat{\delta\bfx} &\gets 0 \\
\bfP &\gets \bfG\bfP\bfG\tr
\end{align}%
%
with the Jacobian
%
\begin{equation}
\bfG = \begin{bmatrix}
\bfI_6 & 0 & 0 \\
0 & \bfI+\hatx{\hat{\frac12\delta\bftheta}} & 0 \\
0 & 0 & \bfI_9
\end{bmatrix}
\end{equation}
%
whose non-trivial term is developed as follows. 
Our goal is to obtain the expression of the new angular error $\delta\bftheta^+$ \wrt the old error $\delta\bftheta$. 
We consider these facts:
\begin{itemize}
\item
The true orientation does not change on error reset, \ie, $\bfq_t^+ \equiv \bfq_t$. This gives:
%
\begin{equation}
\delta\bfq^+\ot\bfq^+ = \delta\bfq\ot \bfq~.
\end{equation}
%
\item
The observed error mean has been injected into the nominal state (see \eqRef{equ:quatErrorInjection} and \eqRef{equ:rotComposition}):
%
\begin{equation}
\bfq^+ = \hat{\delta\bfq} \ot \bfq ~.
\end{equation}
\end{itemize}

Combining both identities we obtain an expression of the new orientation error \wrt the old one and the observed error $\hat{\delta\bfq}$,
%
\begin{equation}
\delta\bfq^+ = \delta\bfq \ot \hat{\delta\bfq}^*  = [\hat{\delta\bfq}^*]_R\cdot\delta\bfq~.
\end{equation}
%
Considering that
%
$\hat{\delta\bfq}^* \approx \begin{bmatrix}
1 \\ -\frac12\hat{\delta\bftheta}
\end{bmatrix}$,
%
the identity above can be expanded as
%
\begin{equation}
\begin{bmatrix}
1 \\ \frac12\delta\bftheta^+
\end{bmatrix}
=
\begin{bmatrix}
1                  & \frac12\hat{\delta\bftheta}\tr \\
-\frac12\hat{\delta\bftheta} & \bfI + \hatx{\frac12\hat{\delta\bftheta}}
\end{bmatrix}
\cdot
\begin{bmatrix}
1 \\ \frac12\delta\bftheta
\end{bmatrix} + \cO(\norm{\delta\bftheta}^2)
\end{equation}
%
which gives one scalar- and one vector- equations,
%
\begin{subequations}
%
\begin{align}
\frac14\hat{\delta\bftheta}\tr\delta\bftheta &= \cO(\norm{\delta\bftheta}^2)\\
\delta\bftheta^+ &= -\hat{\delta\bftheta} + \left(\bfI + \hatx{\frac12\hat{\delta\bftheta}}\right)\delta\bftheta + \cO(\norm{\delta\bftheta}^2)
\end{align}%
\end{subequations}
%
among which the first one is not very informative in that it is only a relation of infinitesimals. 
One can show from the second equation that $\hat{\delta\bftheta}^+ = 0$, which is what we expect from the reset operation. 
The Jacobian is obtained by simple inspection,
%
\begin{equation}
\eqbox{\dpar{\delta\bftheta^+}{\delta\bftheta} = \bfI + \hatx{\frac12\hat{\delta\bftheta}}}~.
\end{equation}

The difference \wrt the local error case is summarized in \tabRef{tab:local_to_global}.


\begin{table*}[tb]
\renewcommand{\arraystretch}{1.7}
\caption{Algorithm modifications related to the definition of the orientation errors.}
\label{tab:local_to_global}
\vspace{1ex}
\centering
\begin{tabular}{|c|c|c|c|}
\hline
Context & Item & local angular error & global angular error \\
\hline
\hline
Error composition & $\bfq_t$ & $\bfq_t = \bfq\ot\delta\bfq$ & $\bfq_t = \delta\bfq\ot\bfq$ \\
\hline
\hline
\multirow{3}{*}{Euler integration} & $\dparil{\delta\bfv^+}{\delta\bftheta}$ & $-\bfR\hatx{\bfa_m-\bfa_b}\Dt$ & $-\hatx{\bfR(\bfa_m-\bfa_b)}\Dt$ \\
&$\dparil{\delta\bftheta^+}{\delta\bftheta}$ & $\bfR\tr\{(\bfomega_m-\bfomega_b)\Dt\}$ & $\bfI$ \\
&$\dparil{\delta\bftheta^+}{\delta\bfomega_b}$ & $-\bfI\Dt$ &  $-\bfR\Dt$ \\
\hline
\hline
Error observation & $
\bfQ_{\delta\bftheta}
%\dparil{(\delta\bfq\otimes\bfq)}{\delta\bftheta}
%\dparil{[\bfq]_L}{\delta\bftheta}
$ & $\frac12\begin{bmatrix}
-q_x &-q_y &-q_z\\
 q_w &-q_z & q_y\\
 q_z & q_w &-q_x\\
-q_y & q_x & q_w\\
\end{bmatrix}$ & $\frac12\begin{bmatrix}
-q_x &-q_y &-q_z\\
 q_w & q_z &-q_y\\
-q_z & q_w & q_x\\
 q_y &-q_x & q_w\\
\end{bmatrix}$ \\
\hline
Error injection &  %$\bfq\gets\bfq\oplus\hat{\delta\bftheta}$  
& $\bfq \gets \bfq \ot \bfq\{\hat{\delta\bftheta}\}$ & $\bfq \gets \bfq\{\hat{\delta\bftheta}\} \ot \bfq$ \\ 
\hline
Error reset & $\dparil{\delta\bftheta^+}{\delta\bftheta}$ & $\bfI - \hatx{\frac12\hat{\delta\bftheta}}$ & $\bfI + \hatx{\frac12\hat{\delta\bftheta}}$ \\
\hline
\end{tabular}
\end{table*}
