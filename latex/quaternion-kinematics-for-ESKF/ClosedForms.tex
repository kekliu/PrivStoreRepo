% !TEX root = kinematics.tex

%%%%%%%%%%%%%%%%%%%%%%%%%%%%%%%%%%%%%%%%%%%%%%%%%%%%%%%%%%%%%%
\section{Closed-form integration methods}
\label{sec:ClosedFormInt}

In many cases it is possible to arrive to a closed-form expression for the integration step. 
We consider now the case of a first-order linear differential equation,
%
\begin{equation}
\dot\bfx(t)=\bfA\tdot\bfx(t)~,
\end{equation}
%
that is, the relation is linear and constant over the interval. 
In such cases, the integration over the interval $[t_n , t_n+\Dt]$ results in
%
\begin{equation}
\bfx_{n+1}= e^{\bfA\tdot\Dt}\bfx_n = \Phi\bfx_n~,
\end{equation}
%
where $\Phi$ is known as the transition matrix. 
The Taylor expansion of this transition matrix is
%
\begin{equation}
\Phi= e^{\bfA\tdot\Dt}=\bfI + \bfA\Dt + \frac{1}{2} \bfA^2\Dt^2 + \frac{1}{3!}\bfA^3\Dt^3 + \dots   = \sum_{k=0}^\infty \frac{1}{k!}\bfA^k\Dt^k ~.
\label{equ:TaylorExp}
\end{equation}


When writing this series for known instances of $\bfA$, it is sometimes possible to identify known series in the result. 
This allows writing the resulting integration in closed form. 
A few examples follow.

%=============================================================
\subsection{Integration of the angular error}
\label{sec:ClosedFormAngle}

For example, consider the angular error dynamics without bias and noise (a cleaned version of Eq.~\eqRef{equ:equat}),
%
\begin{equation}
\dot{\delta\bftheta} = -\hatx\bfomega\delta\bftheta 
\end{equation}
%
Its transition matrix can be written as a Taylor series,
%
%
\begin{align}
\Phi &= e^{-\hatx\bfomega\Dt} \label{equ:phiexp}\\
&= \bfI -\hatx\bfomega\Dt + \frac12\hatx\bfomega^2\Dt^2 - \frac{1}{3!}\hatx\bfomega^3\Dt^3 + \frac{1}{4!}\hatx\bfomega^4\Dt^4 - \dots 
\end{align}%
%
Now defining $\bfomega\Dt \triangleq \bfu \Delta\theta$, the unitary axis of rotation and the rotated angle, and applying \eqRef{equ:prop2}, we can group terms and get
%
%
\begin{align}
\Phi 
&= \bfI -\hatx{\bfu}\Delta\theta + \frac12\hatx{\bfu}^2\Delta\theta^2 - \frac{1}{3!}\hatx{\bfu}^3\Delta\theta^3 + \frac{1}{4!}\hatx{\bfy}^4\Delta\theta^4 - \dots \nonumber\\
&= \bfI - \hatx{\bfu}\left(\Delta\theta - \frac{\Delta\theta^3}{3!} + \frac{\Delta\theta^5}{5!} - \cdots\right) + \hatx{\bfu}^2\left(\frac{\Delta\theta^2}{2!} - \frac{\Delta\theta^4}{4!}  + \frac{\Delta\theta^6}{6!} - \cdots\right) \nonumber\\
&= \bfI - \hatx{\bfu}\sin\Delta\theta + \hatx{\bfu}^2(1-\cos\Delta\theta) ~,
\end{align}%
%
which is a closed-form solution.

\bigskip
This solution corresponds to a rotation matrix, $\Phi=\bfR\{-\bfu\Delta\theta\} = \bfR\{\bfomega\Dt\}\tr$, according to the Rodrigues rotation formula \eqRef{equ:rodrigues}, a result that could be obtained by direct inspection of \eqRef{equ:phiexp} and recalling \eqRef{equ:vectomat}. 
Let us therefore write this as the final closed-form result,
%
\begin{equation}
\eqbox{\Phi=\bfR\{\bfomega\Dt\}\tr}~.
\end{equation}

%=============================================================
\subsection{Simplified IMU example}
\label{sec:IMUexample}

Consider the simplified, IMU driven system with error-state dynamics governed by,
%
\begin{subequations}
%
\begin{align}
\dot{\delta\bfp}   &= \delta\bfv \\
\dot{\delta\bfv}   &= -\bfR\hatx{\bfa}\,\delta\bftheta \\
\dot{\delta\bftheta} &= -\hatx\bfomega\delta\bftheta~,
\end{align}%%
\end{subequations}%
%
where $(\bfa,\bfomega)$ are the IMU readings, and we have obviated gravity and sensor biases. 
This system is defined by the state vector and the dynamic matrix,
%
\begin{equation}
\bfx = \begin{bmatrix}
\delta\bfp \\ \delta\bfv \\ \delta\bftheta
\end{bmatrix}
\qquad
\bfA = \begin{bmatrix}
0 & \bfP_\bfv & 0 \\
0 & 0 & \bfV_\bftheta \\
0 & 0 & \Theta_\bftheta
\end{bmatrix}~.
\end{equation}
%
with
%
%
\begin{align}
\bfP_\bfv &= \bfI \\
\bfV_\bftheta &= -\bfR\hatx{\bfa} \\
\Theta_\bftheta &= -\hatx\bfomega \label{equ:ThetaThetaDef}
\end{align}%


Its integration with a step time $\Dt$ is $\bfx_{n+1}= e^{(\bfA\Dt)}\tdot\bfx_n=\Phi\tdot\bfx_n$. 
The transition matrix $\Phi$ admits a Taylor development \eqRef{equ:TaylorExp}, in increasing powers of $\bfA\Dt$.
We can write a few powers of $\bfA$ to get an illustration of their general form,
%
\begin{equation}
\bfA \!=\! \begin{bmatrix}
0 & \bfP_\bfv & 0 \\
0 & 0 & \bfV_\bftheta \\
0 & 0 & \Theta_\bftheta
\end{bmatrix}
\!,
\bfA^2 \!=\! 
\begin{bmatrix} 
0 & 0 & \bfP_v\bfV_\bftheta \\ 
0 & 0 & \bfV_\bftheta\Theta_\bftheta \\
0 & 0 & \Theta_\bftheta^2
\end{bmatrix}
\!,
\bfA^3 \!=\! 
\begin{bmatrix} 
0 & 0 & \bfP_v\bfV_\bftheta\Theta_\bftheta \\ 
0 & 0 & \bfV_\bftheta\Theta_\bftheta^2 \\
0 & 0 & \Theta_\bftheta^3
\end{bmatrix}
\!,
\bfA^4 \!=\! 
\begin{bmatrix} 
0 & 0 & \bfP_v\bfV_\bftheta\Theta_\bftheta^2 \\ 
0 & 0 & \bfV_\bftheta\Theta_\bftheta^3 \\
0 & 0 & \Theta_\bftheta^4
\end{bmatrix}
\!,
\end{equation}
%
from which it is now visible that, for $k>1$,
%
\begin{equation}
\bfA^{k>1} = 
\begin{bmatrix} 
0 & 0 & \bfP_v\bfV_\bftheta\Theta_\bftheta^{k-2} \\ 
0 & 0 & \bfV_\bftheta\Theta_\bftheta^{k-1} \\
0 & 0 & \Theta_\bftheta^k
\end{bmatrix}
\end{equation}

We can observe that the terms in the increasing powers of $\bfA$ have a fixed part and an increasing power of $\Theta_\bftheta$. 
These powers can lead to closed form solutions, as in the previous section. 


Let us partition the matrix $\Phi$ as follows,
%
\begin{equation}
\Phi = \begin{bmatrix}
\bfI & \Phi_{\bfp\bfv} & \Phi_{\bfp\bftheta} \\
0 & \bfI & \Phi_{\bfv\bftheta} \\
0 & 0 & \Phi_{\bftheta\bftheta}
\end{bmatrix}~,
\end{equation}
%
and let us advance step by step, exploring all the non-zero blocks of $\Phi$ one by one. 

\paragraph{Trivial diagonal terms}
Starting by the two upper terms in the diagonal, they are the identity as shown. 

\paragraph{Rotational diagonal term}
Next is the rotational diagonal term $\Phi_{\bftheta\bftheta}$, relating the new angular error to the old angular error. 
Writing the full Taylor series for this term leads to
%
\begin{equation}
\Phi_{\bftheta\bftheta} = 
\sum_{k=0}^\infty \frac{1}{k!}\Theta_\bftheta^k \Dt^k = \sum_{k=0}^\infty \frac{1}{k!}\hatx{-\bfomega}^k \Dt^k ~, \label{equ:thetaThetaSeries}
\end{equation}
%
which corresponds, as we have seen in the previous section, to our well-known rotation matrix,
%
\begin{equation}
\eqbox{\Phi_{\bftheta\bftheta} = \bfR\{\bfomega\Dt\}\tr}~.
\end{equation}

\paragraph{Position-vs-velocity term}
The simplest off-diagonal term is $\Phi_{\bfp\bfv}$, which is
%
\begin{equation}
\eqbox{\Phi_{\bfp\bfv} = \bfP_\bfv\Dt = \bfI\Dt}~.
\end{equation}

\paragraph{Velocity-vs-angle term}
Let us now move to the term $\Phi_{\bfv, \bftheta}$, by writing its series,
%
\begin{equation}
\Phi_{\bfv\bftheta} = \bfV_\bftheta\Dt 
+ \frac{1}{2}\bfV_\bftheta\Theta_\bftheta\Dt^2 
+ \frac{1}{3!}\bfV_\bftheta\Theta_\bftheta^2\Dt^3
+\cdots
\end{equation}
%
\begin{equation}
\Phi_{\bfv\bftheta} = \Dt\bfV_\bftheta( \bfI 
+ \frac{1}{2}\Theta_\bftheta\Dt 
+ \frac{1}{3!}\Theta_\bftheta^2\Dt^2
+\cdots)
\end{equation}
%
which reduces to
%
\begin{equation}
\Phi_{\bfv\bftheta} = \Dt\bfV_\bftheta\left( \bfI 
+ \sum_{k\ge1}\frac{(\Theta_\bftheta\Dt)^k}{(k+1)!} 
\right)
\end{equation}


At this point we have two options. 
We can truncate the series at the first significant term, obtaining $\Phi_{\bfv\bftheta} = \bfV_\bftheta\Dt$, but this would not be a closed-form. 
See next section for results using this simplified method.
Alternatively, let us factor $\bfV_\bftheta$ out and write
%
\begin{equation}
\Phi_{\bfv\bftheta} = \bfV_\bftheta\Sigma_1
\end{equation}
%
with
%
\begin{equation}
\Sigma_1 = \bfI\Dt 
+ \frac{1}{2}\Theta_\bftheta\Dt^2 
+ \frac{1}{3!}\Theta_\bftheta^2\Dt^3
+\cdots~.
\end{equation}
%
The series $\Sigma_1$ ressembles the series we wrote for $\Phi_{\bftheta\bftheta}$, \eqRef{equ:thetaThetaSeries}, with two exceptions:
\begin{itemize}
\item The powers of $\Theta_\bftheta$ in $\Sigma_1$ do not match with the rational coefficients $\tfrac{1}{k!}$ and with the powers of 
$\Dt$. 
In fact, we remark here that the subindex ``1" in $\Sigma_1$ denotes the fact that one power of $\Theta_\bftheta$ is missing in each of the members.

\item Some terms at the start of the series are missing. 
Again, the subindex ``1" indicates that one such term is missing.
\end{itemize}

The first issue may be solved by applying \eqRef{equ:prop2} to \eqRef{equ:ThetaThetaDef}, which yields the identity
%
\begin{equation}
\Theta_\bftheta
 = \frac{\hatx\bfomega^3}{\norm\bfomega^2}
 = \frac{-\Theta_\bftheta^3 }{\norm\bfomega^2} ~. \label{equ:skewPowerAugment2}
\end{equation}
%
This expression allows us to increase the exponents of $\Theta_\bftheta$ in the series by two, and write, if $\bfomega \neq 0$,
%
\begin{equation}
\Sigma_1 =
  \bfI\Dt 
- \frac{\Theta_\bftheta}{\norm\bfomega^2}\left(\frac12\Theta_\bftheta^2\Dt^2 + \frac1{3!}\Theta_\bftheta^3\Dt^3 + \dots \right) ~,
\end{equation}
%
and $\Sigma_1=\bfI\Dt$ otherwise. 
All the powers in the new series match with the correct coefficients. 
Of course, and as indicated before, some terms are missing. 
This second issue can be solved by adding and substracting the missing terms, and substituting the full series by its closed form. 
We obtain
%
\begin{equation}
\Sigma_1 =
  \bfI\Dt 
- \frac{\Theta_\bftheta}{\norm\bfomega^2}\left(\bfR\{\bfomega\Dt\}\tr - \bfI - \Theta_\bftheta\Dt \right) ~,
\end{equation}
%%
which is a closed-form solution valid if $\bfomega\neq 0$.  
Therefore we can finally write
%
\begin{subequations}
\begin{empheq}
 [left={\Phi_{\bfv\bftheta}=\empheqlbrace},box=\fbox]
 {alignat=2}
 &-\bfR\hatx{\bfa}\Dt   &&  \bfomega\to 0 \\
 &-\bfR\hatx{\bfa} \left(\bfI\Dt 
+ \frac{\hatx\bfomega}{\norm\bfomega^2}\left(\bfR\{\bfomega\Dt\}\tr - \bfI + \hatx\bfomega\Dt \right)
\right)
 &\quad & \bfomega \neq 0
\end{empheq}
\end{subequations}

%\item 
\paragraph{Position-vs-angle term}
Let us finally board the term $\Phi_{\bfp\bftheta}$. 
Its Taylor series is,
%
\begin{equation}
\Phi_{\bfp\bftheta} = 
  \frac{1}{2 }\bfP_\bfv\bfV_\bftheta\Dt^2
+ \frac{1}{3!}\bfP_\bfv\bfV_\bftheta\Theta_\bftheta\Dt^3 
+ \frac{1}{4!}\bfP_\bfv\bfV_\bftheta\Theta_\bftheta^2\Dt^4
+\cdots
\end{equation}
%
We factor out the constant terms and get,
%
\begin{equation}
\Phi_{\bfp\bftheta} = \bfP_\bfv\bfV_\bftheta\ \Sigma_2~,
\end{equation}
%
with
%
\begin{equation}
\Sigma_2 = 
  \frac{1}{2 }\bfI\Dt^2
+ \frac{1}{3!}\Theta_\bftheta\Dt^3 
+ \frac{1}{4!}\Theta_\bftheta^2\Dt^4
+ \cdots ~.
\end{equation}
%
where we note the subindex ``2" in $\Sigma_2$ admits the following interpretation:
%
\begin{itemize}
\item Two powers of $\Theta_\bftheta$ are missing in each term of the series,
\item The first two terms of the series are missing.
\end{itemize}

Again, we use \eqRef{equ:skewPowerAugment2} to increase the exponents of $\Theta_\bftheta$, yielding
%
\begin{equation}
\Sigma_2 = 
  \frac{1}{2}\bfI \Dt^2
- \frac{1}{\norm\bfomega^2}\left(
  \frac{1}{3!}\Theta_\bftheta^3\Dt^3 
+ \frac{1}{4!}\Theta_\bftheta^4\Dt^4
+ \cdots \right)~.
\end{equation}
%
We substitute the incomplete series by its closed form,
%
\begin{equation}
\Sigma_2 =
  \frac12\bfI \Dt^2
- \frac{1}{\norm\bfomega^2}
\left(
  \bfR\{\bfomega\Dt\}\tr 
	- \bfI 
	- \Theta_\bftheta\Dt
	- \frac{1}{2}\Theta_\bftheta^2\Dt^2
\right)~,
\end{equation}
%
which leads to the final result
%
\begin{subequations}
\begin{empheq}[left={\Phi_{\bfp\bftheta}=\empheqlbrace},box=\fbox]{alignat=2}
 & -\bfR\hatx{\bfa} \frac{\Dt^2}{2} && \bfomega\to 0  \\
 & -\bfR\hatx{\bfa} 
\left(
\frac12\bfI \Dt^2
- \frac{1}{\norm\bfomega^2}
\left(
  \bfR\{\bfomega\Dt\}\tr 
	- \sum_{k=0}^2\frac{(-\hatx\bfomega\Dt)^k}{k!}
\right)\right)
 &\quad &\omega\neq 0 
\end{empheq}
\end{subequations}




%=============================================================
\subsection{Full IMU example}
\label{sec:closedFormFullImu}

In order to give means to generalize the methods exposed in the simplified IMU example, we need to examine the full IMU case from a little closer.

Consider the full IMU system \eqRef{equ:efull}, which can be posed as
%
\begin{equation}
\dot{\delta\bfx} = \bfA\delta\bfx + \bfB\bfw~,
\end{equation}
%
whose discrete-time integration requires the transition matrix 
%
\begin{equation}
\Phi=\sum_{k=0}^\infty\frac{1}{k!}\bfA^k\Dt^k = \bfI + \bfA\Dt + \frac12\bfA^2\Dt^2 + \dots~, \label{equ:tranmatseries}
\end{equation}
%
which we wish to compute. 
The dynamic matrix $\bfA$ is block-sparse, and its blocks can be easily determined by examining the original equations \eqRef{equ:efull},

%
\begin{equation}
\bfA = \begin{bmatrix}
  0& \bfP_\bfv&  0&  0&  0&  0\\
  0&  0& \bfV_\bftheta& \bfV_\bfa&  0& \bfV_\bfg\\
  0&  0& \Theta_\bftheta&  0& \Theta_\bfomega
&  0\\
  0&  0&  0&  0&  0&  0\\
  0&  0&  0&  0&  0&  0\\
  0&  0&  0&  0&  0&  0
\end{bmatrix}~. \label{equ:FullIMU_Fmat}
\end{equation}
%

As we did before, let us write a few powers of $\bfA$,
%
%
\begin{align*}
\bfA^2 &= \begin{bmatrix}
  ~~0~~& ~~0~~& ~\bfP_\bfv \bfV_\bftheta~& \bfP_\bfv \bfV_\bfa&          0& \bfP_\bfv \bfV_\bfg\\
  0& 0& ~\bfV_\bftheta \Theta_\bftheta~&    0&      ~\bfV_\bftheta \Theta_\bfomega~&    0\\
  0& 0&  \Theta_\bftheta^2&     0& \Theta_\bftheta \Theta_\bfomega&     0\\
  0& 0&     0&     0&          0&     0\\
  0& 0&     0&     0&          0&     0\\
  0& 0&     0&     0&          0&     0
\end{bmatrix}\\
\bfA^3 &= \begin{bmatrix}
  ~~0~~& ~~0~~& \bfP_\bfv \bfV_\bftheta \Theta_\bftheta& ~0~&             ~\bfP_\bfv \bfV_\bftheta \Theta_\bfomega~& ~~0~~\\
  0& 0&  \bfV_\bftheta \Theta_\bftheta^2&    0&     \bfV_\bftheta \Theta_\bftheta \Theta_\bfomega&    0\\
  0& 0&     \Theta_\bftheta^3&     0& \Theta_\bftheta^2\Theta_\bfomega&     0\\
  0& 0&        0&     0&                    0&     0\\
  0& 0&        0&     0&                    0&     0\\
  0& 0&        0&     0&                    0&     0
\end{bmatrix}\\
\bfA^4 &= \begin{bmatrix}
  ~~0~~& ~~0~~& \bfP_\bfv \bfV_\bftheta \Theta_\bftheta^2 & ~0~&         \bfP_\bfv \bfV_\bftheta\Theta_\bftheta \Theta_\bfomega& ~~0~~\\
  0& 0&    \bfV_\bftheta \Theta_\bftheta^3&    0&  \bfV_\bftheta \Theta_\bftheta^2 \Theta_\bfomega&    0\\
  0& 0&       \Theta_\bftheta^4&     0& \Theta_\bftheta^3\Theta_\bfomega&     0\\
  0& 0&          0&     0&                              0&     0\\
  0& 0&          0&     0&                              0&     0\\
  0& 0&          0&     0&                              0&     0
\end{bmatrix}~.%
\end{align*}%
%
Basically, we observe the following,

\begin{itemize}
\item
The only term in the diagonal of $\bfA$, the rotational term $\Theta_\bftheta$, propagates right and up in the sequence of powers $\bfA^k$. 
All terms not affected by this propagation vanish. 
This propagation afects the structure of the sequence $\{\bfA^k\}$ in the three following aspects:
\item 
The sparsity of the powers of $\bfA$ is stabilized after the 3rd power. 
That is to say, there are no more non-zero blocks appearing or vanishing for powers of $\bfA$ higher than 3.
\item 
The upper-left $3\times 3$ block, corresponding to the simplified IMU model in the previous example, has not changed with respect to that example. 
Therefore, its closed-form solution developed before holds.
\item 
The terms related to the gyrometer bias error (those of the fifth column) introduce a similar series of powers of $\Theta_\bftheta$, which can be solved with the same techniques we used in the simplified example. 
\end{itemize}

We are interested at this point in finding a generalized method to board the construction of the closed-form elements of the transition matrix $\Phi$. 
Let us recall the remarks we made about the series $\Sigma_1$ and $\Sigma_2$,
%
\begin{itemize}
\item The subindex coincides with the lacking powers of $\Theta_\bftheta$ in each of the members of the series.
\item The subindex coincides with the number of terms missing at the beginning of the series.
\end{itemize}

Taking care of these properties, let us introduce the series $\Sigma_n(\bfX,y)$, defined by\footnote{Note that, being $\bfX$ a square matrix that is not necessarily invertible (as it is the case for $\bfX=\Theta_\bftheta$), we are not allowed to rearrange the definition of $\Sigma_n$ with $\Sigma_n=\bfX^{-n}\sum_{k=n}^\infty\frac{1}{k!}(y\bfX)^{k}$.}
%
\begin{equation}
\Sigma_n(\bfX,y) \triangleq \sum_{k=n}^\infty \frac{1}{k!}\bfX^{k-n}y^k = \sum_{k=0}^\infty \frac{1}{(k+n)!}\bfX^{k}y^{\,k+n}= y^n\sum_{k=0}^\infty \frac{1}{(k+n)!}\bfX^k y^k
\end{equation}
%
in which the sum starts at term $n$ and the terms lack $n$ powers of the matrix $\bfX$. 
It follows immediately that $\Sigma_1$ and $\Sigma_2$ respond to 
%
\begin{equation}
\Sigma_n=\Sigma_n(\Theta_\bftheta,\Dt)~,
\end{equation}
%
and that $\Sigma_0 = \bfR\{\bfomega\Dt\}\tr$. 
We can now write the transition matrix \eqRef{equ:tranmatseries} as a function of these series,
%
\begin{equation}
\Phi = \begin{bmatrix}
\bfI & \bfP_\bfv\Dt & \bfP_\bfv\bfV_\bftheta\Sigma_2 &  \tfrac12\bfP_\bfv\bfV_\bfa\Dt^2 & \bfP_\bfv\bfV_\bftheta\Sigma_3\bftheta_\bfomega & \tfrac12\bfP_\bfv\bfV_\bfg\Dt^2 \\
0 & \bfI & \bfV_\bftheta\Sigma_1 &  \bfV_\bfa\Dt & \bfV_\bftheta\Sigma_2\bftheta_\bfomega & \bfV_\bfg\Dt \\
0 & 0 & \Sigma_0 &  0 & \Sigma_1\bftheta_\bfomega & 0 \\
0 & 0 & 0 & \bfI & 0 & 0 \\
0 & 0 & 0 & 0 & \bfI & 0 \\
0 & 0 & 0 & 0 & 0 & \bfI \\
\end{bmatrix}~.\label{equ:fullIMUtranmat}
\end{equation}

Our problem has now derived to the problem of finding a general, closed-form expression for $\Sigma_n$. 
Let us observe the closed-form results we have obtained so far,
%
%
\begin{align}
\Sigma_0 &= \bfR\{\bfomega\Dt\}\tr \\
\Sigma_1 &=
  \bfI\Dt 
- \frac{\Theta_\bftheta}{\norm\bfomega^2}\left(\bfR\{\bfomega\Dt\}\tr - \bfI - \Theta_\bftheta\Dt \right) \\
\Sigma_2 &= 
  \frac12 \bfI\Dt^2
- \frac{1}{\norm\bfomega^2}\left(
  \bfR\{\bfomega\Dt\}\tr - \bfI - \Theta_\bftheta\Dt - \frac12\Theta_\bftheta^2\Dt^2
 \right)~.
\end{align}%

In order to develop $\Sigma_3$, we need to apply the identity \eqRef{equ:skewPowerAugment2} twice (because we lack three powers, and each application of \eqRef{equ:skewPowerAugment2} increases this number by only two), getting
%
\begin{equation}
\Sigma_3 = 
  \frac1{3!}\bfI\Dt^3 
+ \frac{\Theta_\bftheta}{\norm\bfomega^4}
	\left(
		  \frac1{4!}\Theta_\bftheta^4\Dt^4
		+ \frac1{5!}\Theta_\bftheta^5\Dt^5
		+ \dots
	\right)~,
\end{equation}%
%
which leads to
%
\begin{equation}
\Sigma_3 = 
  \frac1{3!}\bfI\Dt^3 
+ \frac{\Theta_\bftheta}{\norm\bfomega^4}
	\left(
		\bfR\{\bfomega\Dt\}\tr - \bfI - \Theta_\bftheta\Dt - \frac12\Theta_\bftheta^2\Dt^2 - \frac1{3!}\Theta_\bftheta^3\Dt^3
	\right)~.
\end{equation}%
%
By careful inspection of the series $\Sigma_0\dots\Sigma_3$, we can now derive a general, closed-form expression for $\Sigma_n$, as follows,
%
\begin{subequations}
\begin{empheq}[left={\Sigma_n=\empheqlbrace},box=\widefbox]{alignat=2}
 & \frac1{n!}\bfI\Dt^n && \bfomega \to 0 \\
 & \bfR\{\bfomega\Dt\}\tr && n = 0 \\
 &  \frac1{n!}\bfI\Dt^n
- \frac{(-1)^{\tfrac{n+1}{2}}\hatx\bfomega}{\norm\bfomega^{n+1}}
	\left(
		\bfR\{\bfomega\Dt\}\tr - \sum_{k=0}^n \frac{(-\hatx\bfomega\Dt)^k}{k!}
	\right) && n \text{ odd}  \\
 &   \frac1{n!}\bfI\Dt^n
+ \frac{(-1)^{\tfrac{n}{2}}}{\norm\bfomega^n}\left(
  \bfR\{\bfomega\Dt\}\tr - \sum_{k=0}^n \frac{(-\hatx\bfomega\Dt)^k}{k!}
	\right) &\quad & n \text{ even} 
\end{empheq}
\end{subequations}

The final result for the transition matrix $\Phi$ follows immediately by substituting the appropriate values of $\Sigma_n,\ n\in\{0,1,2,3\}$, in the corresponding positions of \eqRef{equ:fullIMUtranmat}.

It might be worth noticing that the series now appearing in these new expressions of $\Sigma_n$ have a finite number of terms, and thus that they can be effectively computed. 
That is to say, the expression of $\Sigma_n$ is a closed form as long as $n<\infty$, which is always the case. 
For the current example, we have $n\leq 3$ as can be observed in \eqRef{equ:fullIMUtranmat}.
