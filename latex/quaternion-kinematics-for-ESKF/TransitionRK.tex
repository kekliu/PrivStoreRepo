% !TEX root = kinematics.tex

%%%%%%%%%%%%%%%%%%%%%%%%%%%%%%%%%%%%%%%%%%%%%%%%%%%%%%%%%%%%%%
\section{The transition matrix via Runge-Kutta integration}
\label{sec:TranMatRK}

Still another way to approximate the transition matrix is to use Runge-Kutta integration. 
This might be necessary in cases where the dynamic matrix $\bfA$ cannot be considered constant along the integration interval, \ie,
%
\begin{equation}
\dot\bfx(t)=\bfA(t)\bfx(t)~. \label{equ:RKtran1}
\end{equation}

Let us rewrite the following two relations defining the same system in continuous- and discrete-time. 
They involve the dynamic matrix $\bfA$ and the transition matrix $\Phi$,
%
\begin{align}
\dot\bfx(t) &= \bfA(t)\tdot\bfx(t) \label{equ:ode_dxFx} \\
\bfx(t_n+\tau) &= \Phi(t_n+\tau|t_n)\tdot\bfx(t_n)~.
\end{align}
%
These equations allow us to develop $\dot\bfx(t_n+\tau)$ in two ways as follows (left and right developments, please note the tiny dots indicating the time-derivatives),
%
\begin{align}
\dot{(\Phi(t_n+\tau|t_n)\bfx(t_n))} =& \eqbox{\dot\bfx(t_n+\tau)} =  \bfA(t_n+\tau)\bfx(t_n+\tau) \nonumber \\
\dot\Phi(t_n+\tau|t_n)\bfx(t_n)+\Phi(t_n+\tau|t_n)\dot\bfx(t_n) =&~~~~~~~~~~~~~~~=  \bfA(t_n+\tau)\Phi(t_n+\tau|t_n)\bfx(t_n) \nonumber \\
\dot\Phi(t_n+\tau|t_n)\bfx(t_n) =&& \label{equ:last}
\end{align}%
%
Here, \eqRef{equ:last} comes from noticing that $\dot\bfx(t_n)=\dot\bfx_n=0$, because it is a sampled value. 
Then,
%
\begin{equation}
\dot\Phi(t_n+\tau|t_n) = \bfA(t_n+\tau)\Phi(t_n+\tau|t_n) \label{equ:ode_dPhiFtPhi}
\end{equation}
%
which is the same ODE as \eqRef{equ:ode_dxFx}, now applied to the transition matrix instead of the state vector. 
%
%We know that, in the cases where $\bfA$ is considered constant over the integration interval (\ie, $\bfA(t) = \bfA(t_n)\triangleq\bfA$ for $t_n<t<t_n+\Dt$), the transition matrix can be computed in closed form, $\Phi=e^{\bfA\Dt}$, or approximated by truncation of its Taylor series. 
%If $\bfA(t)$ does not want to be considered constant, then one can attempt numerical integration of \eqRef{equ:ode_dPhiFtPhi} using any high-order runge-Kutta method. 
Mind that, because of the identity $\bfx(t_n)=\Phi_{t_n|t_n}\bfx(t_n)$, the transition matrix at the beginning of the interval, $t=t_n$, is always the identity,
%
\begin{equation}
\Phi_{t_n|t_n} = \bfI~.
\end{equation}

Using RK4 with $f(t,\Phi(t))=\bfA(t)\Phi(t)$, we have
%
\begin{equation}
\Phi\triangleq\Phi(t_n+\Dt|t_n) = \bfI + \frac{\Dt}{6}(\bfK_1+2\bfK_2+2\bfK_3+\bfK_4)
\end{equation}
%
with
%
%
\begin{align}
\bfK_1 &= \bfA(t_n) \\
\bfK_2 &= \bfA\Big(t_n + \frac{1}{2}\Dt \Big)\Big(\bfI + \frac{\Dt}{2} \bfK_1 \Big)\\
\bfK_3 &= \bfA\Big(t_n + \frac{1}{2}\Dt \Big)\Big( \bfI + \frac{\Dt}{2} \bfK_2\Big) \\
\bfK_4 &= \bfA\Big(t_n + \Dt \Big)\Big( \bfI + \Dt \tdot \bfK_3\Big) ~.
\end{align}%

%=============================================================
\subsection{Error-state example}

Let us consider the error-state Kalman filter for the non-linear, time-varying system
%
\begin{equation}
\dot\bfx_t(t) = f(t, \bfx_t(t), \bfu(t))
\end{equation}
%
where $\bfx_t$ denotes the true state, and $\bfu$ is a control input. 
This true state is a composition, denoted by $\oplus$, of a nominal state $\bfx$ and the error state $\delta\bfx$,
%
\begin{equation}
\bfx_t(t) = \bfx(t) \oplus \delta\bfx(t)
\end{equation}
%
where the error-state dynamics admits a linear form which is time-varying depending on the nominal state $\bfx$ and the control $\bfu$, \ie,
%
\begin{equation}
\dot{\delta\bfx} = \bfA(\bfx(t),\bfu(t))\tdot\delta\bfx
\end{equation}
%
that is, the error-state dynamic matrix in \eqRef{equ:RKtran1} has the form $\bfA(t) = \bfA(\bfx(t),\bfu(t))$. 
The dynamics of the error-state transition matrix can be written,
%
\begin{equation}
\dot{\Phi}(t_n+\tau|t_n)=\bfA(\bfx(t),\bfu(t))\tdot\Phi(t_n+\tau|t_n)~.
\end{equation}
%
In order to RK-integrate this equation, we need the values of $\bfx(t)$ and $\bfu(t)$ at the RK evaluation points, which for RK4 are $\{t_n,t_n+\Dt/2,t_n+\Dt\}$. 
Starting by the easy ones, the control inputs $\bfu(t)$ at the evaluation points can be obtained by linear interpolation of the current and last measurements,
%
%
\begin{align}
\bfu(t_n) &= \bfu_n\\
\bfu(t_n+\Dt/2) &= \frac{\bfu_n+\bfu_{n+1}}{2} \\
\bfu(t_n+\Dt) &= \bfu_{n+1}
\end{align}%
%
The nominal state dynamics should be integrated using the best integration practicable. 
For example, using RK4 integration we have,
%
%
\begin{align*}
\bfk_1 &= f(\bfx_n,\bfu_n) \nonumber \\
\bfk_2 &= f(\bfx_n+\frac{\Dt}{2}\bfk_1,\frac{\bfu_n+\bfu_{n+1}}{2}) \nonumber\\
\bfk_3 &= f(\bfx_n+\frac{\Dt}{2}\bfk_2,\frac{\bfu_n+\bfu_{n+1}}{2}) \nonumber\\
\bfk_4 &= f(\bfx_n+\Dt\bfk_3,\bfu_{n+1}) \nonumber\\
\bfk &= (\bfk_1+2\bfk_2+2\bfk_3+\bfk_4)/6 ~,
\end{align*}%
%
which gives us the estimates at the evaluation points,
%
%
\begin{align}
\bfx(t_n) &= \bfx_n \\
\bfx(t_n+\Dt/2) &= \bfx_n+\frac{\Dt}{2}\bfk\\
\bfx(t_n+\Dt) &= \bfx_n+\Dt\,\bfk ~.
\end{align}%
%
We notice here that $\bfx(t_n+\Dt/2)=\frac{\bfx_n+\bfx_{n+1}}{2}$, the same linear interpolation we used for the control. 
This should not be surprising given the linear nature of the RK update.

Whichever the way we obtained the nominal state values, we can now compute the RK4 matrices for the integration of the transition matrix,
%
%
\begin{align*}
\bfK_1 &= \bfA(\bfx_n,\bfu_n) \\
\bfK_2 &= \bfA\Big(\bfx_n+\frac{\Dt}{2}\bfk, \frac{\bfu_n+\bfu_{n+1}}{2}\Big)\Big(\bfI + \frac{\Dt}{2} \bfK_1 \Big)\\
\bfK_3 &= \bfA\Big(\bfx_n+\frac{\Dt}{2}\bfk, \frac{\bfu_n+\bfu_{n+1}}{2}\Big)\Big(\bfI + \frac{\Dt}{2} \bfK_2 \Big)\\
\bfK_4 &= \bfA\Big(\bfx_n+\Dt\bfk, \bfu_{n+1}\Big)\Big(\bfI + \Dt \bfK_3 \Big)\\
\bfK &= (\bfK_1+2\bfK_2+2\bfK_3+\bfK_4)/6 
\end{align*}%
% 
which finally lead to,
%
\begin{equation}
\eqbox{\Phi\triangleq\Phi_{t_n+\Dt|t_n} = \bfI + \Dt\,\bfK}
\end{equation}


