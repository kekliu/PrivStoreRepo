% !TEX root = kinematics.tex

%%%%%%%%%%%%%%%%%%%%%%%%%%%%%%%%%%%%%%%%%%%%%%%%%%%%%%%%%%%%%%
\section{Integration of random noise and perturbations}
\label{sec:IntNoise}

We aim now at giving appropriate methods for the integration of random variables within dynamic systems. 
Of course, we cannot integrate unknown random values, but we can integrate their variances and covariances for the sake of uncertainty propagation. 
This is needed in order to establish the covariances matrices in estimators for systems that are of continuous nature (and specified in continuous time) but estimated in a discrete manner.

Consider the continuous-time dynamic system,
%
\begin{equation}
\dot\bfx = f(\bfx,\bfu,\bfw)~,
\label{equ:contSysWithNoise}
\end{equation}
%
where $\bfx$ is the state vector, $\bfu$ is a vector of  control signals containing noise $\tilde\bfu$, so that the control measurements are $\bfu_m=\bfu+\tilde\bfu$, and $\bfw$ is a vector of random perturbations. 
Both noise and perturbations are assumed white Gaussian processes, specified by,
%
\begin{equation}
\tilde\bfu \sim \cN\{0,\bfU^c\} \quad , \quad \bfw^c\sim\cN\{0,\bfW^c\}~,
\end{equation}
%
where the super-index $\bullet^c$ indicates a continuous-time uncertainty specification, which we want to integrate. 

There exists an important difference between the natures of the noise levels in the control signals, $\tilde\bfu$, and the random perturbations, $\bfw$:
%
\begin{itemize}
\item 
On discretization, the control signals are sampled at the time instants $n\Dt$, having $\bfu_{m,n}\triangleq\bfu_m(n\Dt)=\bfu(n\Dt)+\tilde\bfu(n\Dt)$. 
The measured part is obviously considered constant over the integration interval, \ie, $\bfu_m(t)=\bfu_{m,n}$, and therefore the noise level at the sampling time $n\Dt$ is also held constant,
%
\begin{equation}
\tilde\bfu(t)=\tilde\bfu(n\Dt) = \tilde\bfu_n, \quad n\Dt<t<(n+1)\Dt~. \label{equ:uint}
\end{equation}
%
\item  
The perturbations $\bfw$ are never sampled. 

\end{itemize}
%


As a consequence, the integration over $\Dt$ of these two stochastic processes differs. 
Let us examine it. 

\bigskip
The continuous-time error-state dynamics \eqRef{equ:contSysWithNoise} can be linearized to
%
\begin{equation}
\dot{\delta\bfx} = \bfA\delta\bfx + \bfB\tilde\bfu + \bfC \bfw~, \label{equ:contTime}
\end{equation}
%
with 
%
\begin{equation}
\bfA\triangleq\pjac{f}{\delta\bfx}{\bfx,\bfu_m}\quad,\quad
\bfB\triangleq\pjac{f}{\tilde\bfu}{\bfx,\bfu_m}\quad,\quad
\bfC\triangleq\pjac{f}{\bfw}{\bfx,\bfu_m}~,
\end{equation}
%
and integrated over the sampling period $\Dt$, giving,
%
\begin{align}
\delta\bfx_{n+1} &= \delta\bfx_n 
+ \int_{n\Dt}^{(n+1)\Dt} \left(
\bfA\delta\bfx(\tau) + \bfB\tilde\bfu(\tau) + \bfC\bfw^c(\tau)
\right) d\tau \\
 &= \delta\bfx_n 
+ \int_{n\Dt}^{(n+1)\Dt} \bfA\delta\bfx(\tau)   d\tau
+ \int_{n\Dt}^{(n+1)\Dt} \bfB\tilde\bfu(\tau)   d\tau 
+ \int_{n\Dt}^{(n+1)\Dt} \bfC\bfw^c(\tau) d\tau 
\end{align}
%
which has three terms of very different nature. 
They can be integrated as follows:
%
\begin{enumerate}
\item From \appRef{sec:ClosedFormInt} we know that the dynamic part is integrated giving the transition matrix,
%
\begin{equation}
\delta\bfx_n 
+ \int_{n\Dt}^{(n+1)\Dt} \bfA\delta\bfx(\tau)   d\tau = \Phi\tdot\delta\bfx_n
\end{equation}
%
where $\Phi=e^{\bfA\Dt}$ can be computed in closed-form or approximated at different levels of accuracy.

\item From \eqRef{equ:uint} we have
%
\begin{equation}
\int_{n\Dt}^{(n+1)\Dt} \bfB\tilde\bfu(\tau)   d\tau = \bfB\Dt\tilde\bfu_n
\end{equation}
%
which means that the measurement noise, once sampled, is integrated in a deterministic manner because its behavior inside the integration interval is known.

\item From Probability Theory we know that the integration of continuous white Gaussian noise over a period $\Dt$ produces a discrete white Gaussian impulse $\bfw_n$ described by
%
\begin{equation}
\bfw_n\triangleq\int_{n\Dt}^{(n+1)\Dt}\bfw(\tau)d\tau 
\quad,\quad
\bfw_n \sim \cN\{0,\bfW\} 
\quad,\quad
 \text{with } \bfW = \bfW^c\Dt
\end{equation}
%
We obsereve that, contrary to the measurement noise just above, the perturbation does not have a deterministic behavior inside the integration interval, and hence it must be integrated stochastically.

\end{enumerate}%
%
Therefore, the discrete-time, error-state dynamic system can be written as
%
\begin{equation}
\delta\bfx_{n+1} = \bfF_\bfx\delta\bfx_n + \bfF_\bfu\tilde\bfu_n + \bfF_\bfw \bfw_n \label{equ:discTime}
\end{equation}
%
with  transition, control and perturbation matrices given by
%
\begin{equation}
\bfF_\bfx=\Phi = e^{\bfA\Dt} \quad , \quad \bfF_\bfu=\bfB\Dt \quad , \quad \bfF_\bfw = \bfC \quad , \quad 
\end{equation}
%
%where it is illustrative to appreciate the different contributions of the step time $\Dt$ in the integrated values. The
with noise and perturbation levels defined by
%
\begin{equation}
\tilde\bfu_n\sim\cN\{0,\bfU\} \quad , \quad \bfw_n\sim\cN\{0,\bfW\} \label{equ:discMat}
\end{equation}
%
with 
%
\begin{equation}
\bfU = \bfU^c \quad , \quad \bfW = \bfW^c\Dt~. \label{equ:discCov}
\end{equation}
%
\begin{table*}
\renewcommand{\arraystretch}{1.3}
\caption{Effect of integration on system and covariances matrices.}
\centering
\vspace{1ex}
\begin{tabular}{|c|c|c|}
\hline
Description & Continuous time $t$ & Discrete time $n\Dt$\\
\hline
\hline
state & $\dot\bfx=f^c(\bfx,\bfu,\bfw)$ & $\bfx_{n+1} = f(\bfx_n,\bfu_n,\bfw_n)$ \\
error-state & $\dot{\delta\bfx}=\bfA\delta\bfx+\bfB\tilde\bfu+\bfC\bfw$ & $\delta\bfx_{n+1}=\bfF_\bfx\delta\bfx_n+\bfF_\bfu\tilde\bfu_n+\bfF_\bfw\bfw_n$ \\
\hline
system matrix & $\bfA$ & $\bfF_\bfx=\Phi=e^{\bfA\Dt}$ \\
control matrix & $\bfB$ & $\bfF_\bfu=\bfB\Dt$ \\
perturbation matrix & $\bfC$ & $\bfF_\bfw=\bfC$ \\
\hline
control covariance & $\bfU^c$ & $\bfU=\bfU^c$  \\
perturbation covariance & $\bfW^c$ & $\bfW=\bfW^c\Dt$  \\
\hline
\end{tabular}
\label{tab:IntEffects}
\end{table*}%

These results are summarized in \tabRef{tab:IntEffects}. 
The prediction stage of an EKF would propagate the error state's mean and covariances matrix according to
%
\begin{align}
\hat{\delta\bfx}_{n+1} &= \bfF_\bfx \hat{\delta\bfx}_n \\
\bfP_{n+1} &= \bfF_\bfx\bfP_n\bfF_\bfx\tr + \bfF_\bfu\bfU\bfF_\bfu\tr + \bfF_\bfw\bfW\bfF_\bfw\tr \nonumber\\
&= e^{\bfA\Dt}\bfP_n(e^{\bfA\Dt})\tr + \Dt^2\bfB\bfU^c\bfB\tr + \Dt\bfC\bfW^c\bfC\tr \label{equ:NoisePertCovUpdate}
\end{align}

It is important and illustrative here to observe the different effects of the integration interval, $\Dt$, on the three terms of the covariance update \eqRef{equ:NoisePertCovUpdate}: the dynamic error term is exponential, the measurement error term is quadratic, and the perturbation error term is linear. 


%=============================================================
\subsection{Noise and perturbation impulses}
\label{sec:pertImpulses}

One is oftentimes confronted (for example when reusing existing code or when interpreting other authors' documents) with EKF prediction equations of a simpler form than those that we used here, namely,
%
\begin{equation}
\bfP_{n+1} = \bfF_\bfx\bfP_n\bfF_\bfx\tr + \bfQ~.
\end{equation}
%
This corresponds to the general discrete-time dynamic system,
%
\begin{equation}
\delta\bfx_{n+1} = \bfF_\bfx\delta\bfx_n+\bfi
\end{equation}
%
where
%
\begin{equation}
\bfi \sim \cN\{0,\bfQ\}
\end{equation}
%
is a vector of random (white, Gaussian) impulses that are directly added to the state vector at time $t_{n+1}$. 
The matrix $\bfQ$ is simply considered the impulses covariances matrix. 
From what we have seen, we should compute this covariances matrix as follows,
%
\begin{equation}
\bfQ = \Dt^2\,\bfB\,\bfU^c\,\bfB\tr + \Dt\,\bfC\,\bfW^c\,\bfC\tr~.
\end{equation}

In the case where the impulses do not affect the full state, as it is often the case, the matrix $\bfQ$ is not full-diagonal and may contain a significant amount of zeros. 
One can then write the equivalent form
%
\begin{equation}
\delta\bfx_{n+1} = \bfF_\bfx\,\delta\bfx_n+\bfF_\bfi\,\bfi
\end{equation}
%
with
%
\begin{equation}
\bfi \sim \cN\{0,\bfQ_\bfi\}~,
\end{equation}
%
where the matrix $\bfF_\bfi$ simply maps each individual impulse to the part of the state vector it affects to. 
The associated covariance $\bfQ_\bfi$ is then smaller and full-diagonal. 
Please refer to the next section for an example. 
In such case the ESKF time-update becomes
%
\begin{align}
\hat{\delta\bfx}_{n+1} &= \bfF_\bfx\,\hat{\delta\bfx}_n \\
\bfP_{n+1} &= \bfF_\bfx\,\bfP_n\,\bfF_\bfx\tr + \bfF_\bfi\,\bfQ_\bfi\,\bfF_\bfi\tr~.
\end{align}
%


Obviously, all these forms are equivalent, as it can be seen in the following double identity for the general perturbation $\bfQ$,
%
\begin{equation}
\bfF_\bfi\,\bfQ_\bfi\,\bfF_\bfi\tr = \eqbox{\bfQ} =  \Dt^2\,\bfB\,\bfU^c\,\bfB\tr + \Dt\,\bfC\,\bfW^c\,\bfC\tr~.
\end{equation}



%=============================================================
\subsection{Full IMU example}

We study the construction of an error-state Kalman filter for an IMU. 
The error-state system is defined in \eqRef{equ:efull} and involves a nominal state $\bfx$, an error-state $\delta\bfx$, a noisy control  signal $\bfu_m=\bfu+\tilde\bfu$ and a perturbation $\bfw$, specified by,
%
\begin{equation}
\bfx = \begin{bmatrix}
\bfp \\
\bfv \\
\bfq \\
\bfa_b \\
\bfomega_b \\
\bfg
\end{bmatrix}
\quad , \quad
\delta\bfx = \begin{bmatrix}
\delta\bfp \\
\delta\bfv \\
\delta\bftheta \\
\delta\bfa_b \\
\delta\bfomega_b \\
\delta\bfg
\end{bmatrix}
\quad , \quad
\bfu_m=\begin{bmatrix}
\bfa_m \\ \bfomega_m
\end{bmatrix} 
\quad , \quad
\tilde\bfu=\begin{bmatrix}
\tilde\bfa \\ \tilde\bfomega
\end{bmatrix} 
\quad , \quad
\bfw = \begin{bmatrix}
\bfa_w \\ \bfomega_w
\end{bmatrix}
\end{equation}

In a model of an IMU like the one we are considering throughout this document, the control noise corresponds to the additive noise in the IMU measurements. 
The perturbations affect the biases, thus producing their random-walk behavior. 
The dynamic, control and perturbation matrices are (see  \eqRef{equ:contTime}, \eqRef{equ:FullIMU_Fmat} and \eqRef{equ:efull}),
%
\begin{equation}
\label{equ:IMU_ABC}
\bfA = \begin{bmatrix}
  0& \bfP_\bfv&  0&  0&  0&  0\\
  0&  0& \bfV_\bftheta& \bfV_\bfa&  0& \bfV_\bfg\\
  0&  0& \Theta_\bftheta&  0& \Theta_\bfomega&  0\\
  0&  0&  0&  0&  0&  0\\
  0&  0&  0&  0&  0&  0\\
  0&  0&  0&  0&  0&  0
\end{bmatrix}
\quad,\quad
\bfB = \begin{bmatrix}
0 & 0 \\
-\bfR & 0 \\
0 & -\bfI \\
0 & 0 \\
0 & 0 \\
0 & 0 \\
\end{bmatrix}
\quad,\quad
\bfC = \begin{bmatrix}
0 & 0 \\
0 & 0 \\
0 & 0 \\
\bfI & 0 \\
0 & \bfI \\
0 & 0 \\
\end{bmatrix}
\end{equation}

In the regular case of IMUs with accelerometer and gyrometer triplets of the same kind on the three axes, noise and perturbations are isotropic. 
Their standard deviations are specified as scalars as follows
%
\begin{equation}
\sigma_{\tilde\bfa}\ [m/s^2] \quad,\quad \sigma_{\tilde\bfomega} \  [rad/s] \quad,\quad \sigma_{\bfa_w}\ [m/s^2\sqrt{s}] \quad,\quad \sigma_{\bfomega_w}\ [rad/s\sqrt{s}]
\end{equation}
%
and their covariances matrices are purely diagonal, giving
%
\begin{equation}
\bfU^c=\begin{bmatrix}
\sigma_{\tilde\bfa}^2\bfI & 0 \\ 0 & \sigma_{\tilde\bfomega}^2\bfI  
\end{bmatrix} \qquad ,\qquad
\bfW^c=\begin{bmatrix}
\sigma_{\bfa_w}^2\bfI & 0 \\ 0 & \sigma_{\bfomega_w}^2\bfI  
\end{bmatrix} ~.
\end{equation}

The system evolves with sampled measures at intervals $\Dt$, following \eqsRef{equ:discTime}{equ:discCov}, where the transition matrix $\bfF_\bfx=\Phi$ can be computed in a number of ways -- see previous appendices. 

%%-------------------------------------------------------------
\subsubsection{Noise and perturbation impulses}
%\label{sec:pertImpulsesIMU}

%\TODO{Describe first the general case, and then the isotropic case}

In the case of a perturbation specification in the form of impulses $\bfi$, we can re-define our system as follows,
%
\begin{equation}
\delta\bfx_{n+1} = \bfF_\bfx(\bfx_n, \bfu_m)\tdot \delta\bfx_n+\bfF_\bfi\tdot\bfi
\end{equation}
%
with the nominal-state, error-state, control, and impulses vectors defined by,
%
\begin{equation}
\bfx=\begin{bmatrix}\bfp\\\bfv\\\bfq\\\bfa_b\\\bfomega_b\\\bfg\end{bmatrix} \quad, \quad
\delta\bfx=\begin{bmatrix}\delta\bfp\\\delta\bfv\\\delta\bftheta\\\delta\bfa_b\\\delta\bfomega_b\\\delta\bfg\end{bmatrix} \quad, \quad
\bfu_m = \begin{bmatrix}
\bfa_m \\
\bfomega_m
\end{bmatrix} 
\quad,  \quad
\bfi = \begin{bmatrix}
\bfv_\bfi \\
\bftheta_\bfi \\
\bfa_\bfi \\
\bfomega_\bfi
\end{bmatrix}~,
\end{equation}
%
the transition and perturbations matrices defined by,
%
\begin{equation}
\bfF_\bfx = \Phi = e^{\bfA\Dt}
\qquad,\qquad
\bfF_\bfi 
%= \begin{bmatrix}
%\bfB & \bfC
%\end{bmatrix} 
= \begin{bmatrix}
 0 &  0 & 0 & 0 \\
 \bfI &  0 & 0 & 0 \\
 0 &  \bfI & 0 & 0 \\
 0 &  0 & \bfI & 0 \\
 0 & 0 & 0 & \bfI \\
 0 &  0 & 0 & 0 
\end{bmatrix} ~,
\end{equation}
%
and the impulses variances specified by
%
\begin{equation}
\bfi \sim \cN\{0,\bfQ_\bfi\}
\quad,\quad
\bfQ_\bfi = \begin{bmatrix}
\sigma_{\tilde\bfa}^2\Dt^2\bfI &&0&\\
& \sigma_{\tilde\bfomega}^2\Dt^2\bfI &&\\
&& \sigma_{\bfa_w}^2\Dt\bfI &\\
&0&& \sigma_{\bfomega_w}^2\Dt\bfI 
\end{bmatrix}~.
\end{equation}



The trivial specification of $\bfF_\bfi$ may appear surprising given especially that of $\bfB$ in \eqRef{equ:IMU_ABC}. 
What happens is that the errors are defined isotropic in $\bfQ_\bfi$, and therefore $-\bfR\sigma^2\bfI(-\bfR)\tr = \sigma^2\bfI$ and $-\bfI\sigma^2\bfI(-\bfI)\tr = \sigma^2\bfI$, leading to the expression given for $\bfF_\bfi$. 
This is not possible when considering non-isotropic IMUs, where a proper Jacobian $\bfF_\bfi=\begin{bmatrix}
\bfB & \bfC
\end{bmatrix}$ should be used together with a proper specification of $\bfQ_\bfi$.

%\begin{bmatrix}
% 0 &  0 & 0 & 0 \\
% \bfI &  0 & 0 & 0 \\
% 0 &  \bfI & 0 & 0 \\
% 0 &  0 & \bfI & 0 \\
% 0 & 0 & 0 & \bfI \\
% 0 &  0 & 0 & 0 
%\end{bmatrix}

\bigskip
We can of course use full-state perturbation impulses,
%
\begin{equation}
\delta\bfx_{n+1} = \bfF_\bfx(\bfx_n, \bfu_m)\tdot \delta\bfx_n+\bfi
\end{equation}
%
with
%
\begin{equation}
\bfi = \begin{bmatrix}
0 \\
\bfv_\bfi \\
\bftheta_\bfi \\
\bfa_\bfi \\
\bfomega_\bfi \\
0
\end{bmatrix}
\quad,\quad
\bfi \sim \cN\{0,\bfQ\}
\quad,\quad
\bfQ = \begin{bmatrix}
0 & \\
& \sigma_{\tilde\bfa}^2\Dt^2\bfI &&&0&\\
&& \sigma_{\tilde\bfomega}^2\Dt^2\bfI &&\\
&&& \sigma_{\bfa_w}^2\Dt\bfI &\\
&0&&& \sigma_{\bfomega_w}^2\Dt\bfI \\
&&&&&0
\end{bmatrix}~.
\end{equation}



\bigskip
\bigskip
\bigskip

Bye bye.

