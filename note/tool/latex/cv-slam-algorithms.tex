\section{SLAM主流算法}
\subsection{匹配方法}
\subsubsection{特征点/线/面匹配(激光和视觉)}
\paragraph{ICP}
\paragraph{loam}
\subsubsection{直接法(视觉)}
\paragraph{光流法}
\paragraph{直接法}
\subsubsection{概率格网(激光)}
\subsection{优化策略}
\subsubsection{滤波器方法}
\subsubsection*{卡尔曼滤波: KF}
已知状态方程和观测方程
\begin{equation}
  \left\{
  \begin{array}{lr}
    x_t = F x_{t-1} + B u_{t-1} + w \\
    z_t = H x_t + v
  \end{array}
  \right.
\end{equation}
\par 其中,$x_t$是系统状态量(如位置、速度、航向),$u_{t-1}$是控制量(如角速度、力),$F$是状态转移矩阵,$B$是输入控制矩阵,$w$是过程噪声,$z_t$是量测量,$H$是状态向量到量测量的变换矩阵,$v$是量测噪声。请注意,上面的$F,B,w,H,v$都有下标t或t-1。过程噪声$w$和量测噪声$v$满足零均值高斯分布:
\begin{equation}
  w \sim N(0,R),v \sim N(0,Q)
\end{equation}
\par 进一步得到下面的方程组:
\begin{equation}
  \left\{
  \begin{split}
    P(x_t)
    &=N(\overline{x}_t,\overline{P}_t)
    =N(F \hat{x}_{t-1} + B u_{t-1}, F \hat{P}_{t-1} F^T + R) \\
    P(z_t|x_t) &=N(H x_t,Q) \\
    P(x_t|z_t) &= \frac{P(z_t|x_t) P(x_t)}{P(z_t)} \propto P(z_t|x_t) P(x_t)
  \end{split}
  \right.
\end{equation}
\par 其中,第三个等式中的$P(z_t|x_t)$叫似然,$P(x_t)$叫先验。在只有观测方程、没有状态方程时,只能使用MLE(最大似然估计);在有状态方程后,可以借助先验概率$P(x_t)$,通过MAP(最大后验概率)估计$x_t$。
\begin{equation}
  N(\hat{x}_t,\hat{P}_t) \propto N(\overline{x}_t,\overline{P}_t) \cdot N(H x_t,Q)
\end{equation}
\par 考虑等式两边的指数部分,可以得到
\begin{equation}
  (x_t-\hat{x}_t)^T\hat{P}_t^{-1}(x_t-\hat{x}_t)
  =
  (z_t-H x_t)^T Q^{-1} (z_t-H x_t)
  +
  (x_t-\hat{x}_t)^T \overline{P}_t^{-1} (x_t-\hat{x}_t)
\end{equation}
\par 比较等式两端$x_t$的一次项和二次项系数,可以得到卡尔曼滤波器的结果:
\par {a. 预测公式}
\begin{equation}
  \overline{x}_t=F\hat{x}_{t-1} + B u_{t-1}
\end{equation}
\begin{equation}
  \overline{P}_t=F\hat{P}_{t-1}F^T+R
\end{equation}
\par {b. 更新公式}
\begin{equation}
  K=\overline{P}_t H^T(H \overline{P}_t H^T + Q)^{-1}
\end{equation}
\begin{equation}
  \hat{x}_t=\overline{t}_t+K(z_t-H \overline{x}_t)
\end{equation}
\begin{equation}
  \hat{P}_t=(I-K H)\overline{P}_t
\end{equation}
\subsubsection*{扩展卡尔曼滤波: EKF}
\subsubsection*{无迹卡尔曼滤波: UKF}
\subsubsection*{粒子滤波: PF}
\subsection{优化方法}
\subsubsection{关键帧BA}
\subsubsection{图优化}






