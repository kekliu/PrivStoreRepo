\section{VINS预积分推导}
\subsection{记号说明}
\subsubsection{上下标}
\par b = body frame = imu frame
\par w = world frame

\subsection{IMU noise model}
\begin{equation}
	\begin{split}
		\hat{a}_t &= a_t + R_w^t g^w + b_{a_t} + n_a\\
		&= R_w^t (a_t^w+g^w)+b_{a_t}+n_{a}\\
		\hat{\omega}_t &= \omega_t+b_{gt}+n_{g}
	\end{split}
\end{equation}
$\hat{a}_t$ is in IMU frame and $a_t^w$ is in world frame.

\subsection{Preintegration on continuous time}
In world frame, kinetic equations can be written as follows:
\begin{equation}
	\begin{split}
		p_{b_{k+1}}^w & = p_{b_k}^w + v_{b_k}^w \Delta t_k +\iint_{t\in [t_k,t_{k+1}]} (R^w_t(\hat{a}_t-b_{a_t}-n_a)-g^w) dt^2 \\
		v_{b+1}^w & = v_{b_k}^w + \int_{t\in [t_k,t_{k+1}]} (R^w_t(\hat{a}_t-b_{a_t}-n_a)-g^w) dt \\
		q_{b_{k+1}}^w & = q_{b_k}^w \otimes \int_{t\in [t_k,t_{k+1}]} \frac{1}{2} \Omega(\hat{w}_t-b_{g_t}-n_g)q_t^{b_k} dt
	\end{split}
\end{equation}
\par In IMU frame w.r.t. previous time,
\begin{equation}
	\begin{split}
		R_w^{b_k}p_{b_{k+1}}^w &= R_w^{b_k}( p_{b_k}^w + v_{b_k}^w\Delta{t} - \frac12g^w\Delta{t}^2) + \alpha_{b_{k+1}}^{b_k} \\
		R_w^{b_k}v_{b_{k+1}}^w &= R_w^{b_k}( v_{b_k}^w - g^w\Delta{t} ) + \beta_{b_{k+1}}^{b_k} \\
		q_{b_{k+1}}^w &= q_{b_k}^w \otimes \gamma_{b_{k+1}}^{b_k}
	\end{split}
\end{equation}
\par where contains preintegration quantity $\alpha_{b_{k+1}}^{b_k}$, $\beta_{b_{k+1}}^{b_k}$ and $\gamma_{b_{k+1}}^{b_k}$.
\begin{equation}
	\begin{split}
		\alpha_{b_{k+1}}^{b_k} &= \iint_{t\in[t_k,t_{k+1}]}R_t^{b_k}(\hat{a}_t-b_{a_t}-n_a)dt^2 \\
		\beta_{b_{k+1}}^{b_k} &= \int_{t\in[t_k,t_{k+1}]}R_t^{b_k}(\hat{a}_t-b_{a_t}-n_a)dt \\
		\gamma_{b_{k+1}}^{b_k} &= \int_{t\in[t_k,t_{k+1}]}\frac12\Omega(\hat{\omega}_t-b_{g_t}-n_g)q_tdt
	\end{split}
\end{equation}
\par As we see preintegration quantity is only related with IMU bias $b_{a_t}$ and $b_{g_t}$.

In discrete time, preintegration quantity can be iterated as Eq. \ref{eq:discrete-time-preintegration}. Noise $n_{a_t}$ and $n_{g_t}$ are ignored in the equation because they can not be predicted and their integration can be treated as zero during small interval (in fact the integration of is small random walk). $\alpha_i^{b_k}$, $\beta_i^{b_k}$, $\gamma_i^{b_k}$ will be initialized to be 0 or identity rotation respectively.
\begin{equation} \label{eq:discrete-time-preintegration}
	\begin{split}
		\alpha_{i+1}^{b_k} &= \alpha_i^{b_k} + \frac12(\beta_i^{b_k}+\beta_{i+1}^{b_k})\delta{t} \\
		\beta_{i+1}^{b_k} &= \beta_i^{b_k} + \frac12\left[\gamma_i^{b_k}(\hat{a}_i-b_{a_i})+\gamma_{i+1}^{b_k}(\hat{a}_{i+1}-b_{a_i})\right]\delta{t} \\
		\gamma_{i+1}^{b_k} &= \gamma_i^{b_k} \otimes
		\begin{bmatrix}
			1 \\ \frac12(\frac{\hat{w}_i+\hat{w}_{i+1}}2-b_{g_i})\delta{t}
		\end{bmatrix}
	\end{split}
\end{equation}
Applying Eq \ref{eq:discrete-time-preintegration} we get preintegration quantity between two consecutive frames: $\alpha_{b_{k+1}}^{b_k}$, $\beta_{b_{k+1}}^{b_k}$ and $\gamma_{b_{k+1}}^{b_k}$. IMU bias between consecutive frame changes quite small so it can be treated as constant between frame $i$ and frame $i+1$.
So preintegration quantity can be expanded in first-order-taylor:
\begin{equation}
	\begin{split}
		\alpha_{b_{k+1}}^{b_k} &= \hat{\alpha}_{b_{k+1}}^{b_k}+J\delta{b_{a_k}}+J\delta{b_{g_k}} \\
		\beta_{b_{k+1}}^{b_k} &= \hat{\beta}_{b_{k+1}}^{b_k}+J\delta{b_{a_k}}+J\delta{b_{g_k}} \\
		\theta_{b_{k+1}}^{b_k} &= tdb
	\end{split}
\end{equation}
