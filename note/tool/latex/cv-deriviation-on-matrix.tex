\section{矩阵求导表示}
\subsection{布局}
矩阵求导有两种布局,分子布局(numerator layout)和分母布局(denominator layout)。
分子布局的导数矩阵布局与分母相同(别问我为啥这样起名字,我也不知道)。
/par 如无特别说明,下面应用的都是分子布局。
\subsection{基本求导规则}
\subsubsection{基本类型}
\par 已知标量$s$、$m$维向量$\boldsymbol{x}$、$n$维向量$\boldsymbol{y}$、$s\times t$维矩阵$\boldsymbol{A}$,则有以下基本求导形式:

\par 向量对标量求导$m\times 1$
\begin{equation}
  \frac{\partial\boldsymbol{x}}{\partial s}
\end{equation}

\par 标量对向量求导$1\times m$
\begin{equation}
  \frac{\partial s}{\partial\boldsymbol{x}}
\end{equation}

\par 向量对向量求导$m\times n$
\begin{equation}
  \frac{\partial\boldsymbol{x}}{\partial\boldsymbol{y}}
\end{equation}

\par 矩阵对标量求导$s\times t$
\begin{equation}
  \frac{\partial\boldsymbol{A}}{\partial s}
\end{equation}

\par 标量对矩阵求导$t\times s$(??)
\begin{equation}
  \frac{\partial s}{\partial\boldsymbol{A}}
\end{equation}

\subsubsection{推论}
\begin{equation}
  \frac{\partial \boldsymbol{Ax}}{\partial \boldsymbol{y}}
  =
  A \frac{\partial \boldsymbol{x}}{\partial \boldsymbol{y}}
\end{equation}
